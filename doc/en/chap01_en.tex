% !TEX root = userguide_en.tex
%----------------------------------------------------------
\chapter{What is ${\HPhi}$?}
\label{Ch:whatishphi}
%----------------------------------------------------------
%----------------------------------------------------------
%----------------------------------------------------------
%----------------------------------------------------------
\section{What is $\HPhi$?}
Comparison between experimental observation and theoretical analysis is a crucial step
in condensed-matter physics research. The tTemperature dependence of specific heat and
magnetic susceptibility, for example, has been studied to extract the nature of low energy
excitations of and magnetic interactions between electrons, respectively, through comparison
with theories such as Landau's Fermi liquid theory and the Curie-Weiss law.

For the flexible and quantitative comparison of theoretical and  experimental data,
the exact diagonalization
approach~\cite{Dagotto} is one of the most reliable numerical tools that requires no approximation or
inspiration of genius. For the last few decades, a numerical diagonalization package for quantum
spin Hamiltonians, TITPACK, developed by Prof. Hidetoshi Nishimori of Tokyo Institute of Technology,
has been widely used in the condensed-matter physics community. Nevertheless, limited
computational resources have hindered the ability of non-expert users to apply the package to
quantum systems with a large number of electrons or spins.

In contrast, the recent and rapid development of a parallel computing infrastructure has opened up new
avenues for user-friendly larger scale diagonalizations up to 18-site Hubbard clusters
or 36 $S$=$1/2$ quantum spins. In addition, recent advances in quantum statistical mechanics ~\cite{Imada1986,FTLanczos,Hams,Sugiura2012}
allow the finite temperature properties of quantum many-body systems
to be calculated at computational costs similar to those of the calculations of ground state properties,
which also allows theoretical results for the temperature dependence
of, for example, specific heat and magnetic susceptibility,
to be compared with experimental results quantitatively~\cite{Yamaji2014}.
To utilize the parallel computing infrastructure with narrow bandwidth and distributed-memory
architectures, efficient, user-friendly, and highly parallelized diagonalization packages are highly desirable.

$\HPhi$, a flexible diagonalization package for solving quantum lattice Hamiltonians,
has been developed as a descendant of the pioneering package TITPACK.
The Lanczos method for calculations of the ground state and a few excited states properties,
as well as finite temperature calculations based on thermal pure quantum states~\cite{Sugiura2012},
are implemented in
the $\HPhi$ package, with an easy-to-use and flexible user interface.
By using $\HPhi$, you can analyze a wide range of quantum lattice Hamiltonians including
simple Hubbard and Heisenberg models, multi-band extensions of the Hubbard model,
exchange couplings that break the SU(2) symmetry of quantum spins, such as Dzyaloshinskii-Moriya
and Kitaev interactions, and Kondo lattice models describing itinerant electrons coupled with
quantum spins. $\HPhi$ calculates a variety of physical quantities, such as internal energy at zero temperature or finite temperatures, temperature dependence of specific heat, and charge/spin structure factors.
A broad spectrum of users including experimental scientists is cordially welcome.

\subsection{License}
The distribution of the program package and the source codes for ${\mathcal H \Phi}$ follow GNU General Public License version 3 (GPL v3). 
We kindly ask you to cite the article:
Mitsuaki Kawamura, Kazuyoshi Yoshimi, Takahiro Misawa, Youhei Yamaji, Synge Todo, and Naoki Kawashima, 
Comp. Phys. Commun. 217 (2017) 180-192. in publications that include results obtained using this software.


\subsection{Copyright}
{\it \copyright 2015- The University of Tokyo.} {\it  All rights reserved.}
%{\it \copyright 2015- T. Misawa, K. Yoshimi, M. Kawamura, Y. Yamaji, S. Todo and N. Kawashima.} {\it  All rights reserved.}\\
This software was developed with the support of ``{\it Project for advancement of software usability in materials science}" of The Institute for Solid State Physics, The University of Tokyo. 
%The copyright of this software belongs to The University of Tokyo.
\subsection{Contributors}
\label{subsec:contributors}
This software was developed by the following contributors.
\begin{itemize}

\item{ver.3.1 (released on 2018/9/3)}
\begin{itemize}
	\item{\bf Developers}
	\begin{itemize}
	\item{Takahiro Misawa \\(The Institute for Solid State Physics, The University of Tokyo)}
	\item{Kazuyoshi Yoshimi\\ (The Institute for Solid State Physics, The University of Tokyo)}
	\item{Mitsuaki Kawamura\\ (The Institute for Solid State Physics, The University of Tokyo)}
	\item{Kota Ido\\ (Department of Applied Physics, The University of Tokyo)}
	\item{Youhei Yamaji\\ (Department of Applied Physics, The University of Tokyo)}
	\item{Synge Todo\\ (Department of Physics, The University of Tokyo)}
	\item{Yusuke Konishi\\ (Academeia Co., Ltd.)}
	\end{itemize}
	\item{\bf Project coordinator}
	\begin{itemize}
	\item{Naoki Kawashima\\ (The Institute for Solid State Physics, The University of Tokyo)}
	\end{itemize}
\end{itemize}


\item{ver.3.0 (released on 2017/12/25)}
\begin{itemize}
	\item{\bf Developers}
	\begin{itemize}
	\item{Takahiro Misawa \\(The Institute for Solid State Physics, The University of Tokyo)}
	\item{Kazuyoshi Yoshimi\\ (The Institute for Solid State Physics, The University of Tokyo)}
	\item{Mitsuaki Kawamura\\ (The Institute for Solid State Physics, The University of Tokyo)}
	\item{Kota Ido\\ (Department of Applied Physics, The University of Tokyo)}
	\item{Youhei Yamaji\\ (Department of Applied Physics, The University of Tokyo)}
	\item{Synge Todo\\ (Department of Physics, The University of Tokyo)}
	\end{itemize}
	\item{\bf Project coordinator}
	\begin{itemize}
	\item{Naoki Kawashima\\ (The Institute for Solid State Physics, The University of Tokyo)}
	\end{itemize}
\end{itemize}

\item{ver.2.0 (released on 2017/4/11)}
\item{ver.1.2 (released on 2016/11/14)}
\item{ver.1.1 (released on 2016/5/13)}
\item{ver.1.0 (released on 2016/4/5)}
\begin{itemize}
	\item{\bf Developers}
	\begin{itemize}
	\item{Takahiro Misawa \\(Department of Applied Physics, The University of Tokyo)}
	\item{Kazuyoshi Yoshimi\\ (The Institute for Solid State Physics, The University of Tokyo)}
	\item{Mitsuaki Kawamura\\ (The Institute for Solid State Physics, The University of Tokyo)}
	\item{Youhei Yamaji\\ (Department of Applied Physics, The University of Tokyo)}
	\item{Synge Todo\\ (Department of Physics, The University of Tokyo)}
	\end{itemize}
	\item{\bf Project coordinator}
	\begin{itemize}
	\item{Naoki Kawashima\\ (The Institute for Solid State Physics, The University of Tokyo)}
	\end{itemize}
\end{itemize}
\end{itemize}

\section{Operating environment}
 $\HPhi$ was tested on the following platforms:

\begin{itemize}
\item The supercomputer system-B ``sekirei'' in ISSP
\item Fujitsu Fx-10 and K computer
\item Linux PC + Intel compiler
\item Linux PC + gcc
\item Mac + gcc.
\end{itemize}
