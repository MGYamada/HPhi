% !TEX root = userguide_en.tex
%----------------------------------------------------------
\chapter{What is ${\HPhi}$?}
\label{Ch:whatishphi}
%----------------------------------------------------------
%----------------------------------------------------------
%----------------------------------------------------------
%----------------------------------------------------------
\section{What is $\HPhi$?}
Comparison between experimental observation and theoretical analysis is a crucial step
in condensed-matter physics research. Temperature dependence of specific heat and
magnetic susceptibility, for example, have been studied to extract nature of low energy
excitations of and magnetic interactions among electrons, respectively, through comparison
with theories such as Landau's Fermi liquid theory and Curie-Weiss law.

For the flexible and quantitative comparison with experimental data, an exact diagonalization
approach~\cite{Dagotto} is one of the most reliable numerical tools without any approximation or
inspiration of genius. For last few decades, a numerical diagonalization package for quantum
spin hamiltonians, TITPACK developed by Prof. Hidetoshi Nishimori in Tokyo Institute of Technology,
has been widely used in the condensed-matter physics community. Nevertheless, limitation of
computational resources had hindered the non-expert users from applying the package to
quantum systems with large number of electrons or spins.

In contrast, recent and rapid development of parallel computing infrastructure opens up new
avenues for user-friendly larger scale diagonalizations up to 18 site Hubbard clusters
or 36 $S$=$1/2$ quantum spins. In addition, recent advances in quantum statistical mechanics ~\cite{Imada1986,FTLanczos,Hams,Sugiura2012}
enable us to calculate finite temperature properties of quantum many-body systems
with computational costs similar to calculations of ground state properties,
which also enables us to compare theoretical results for temperature dependence
of, for example, specific heat and magnetic susceptibility with experimental results quantitatively~\cite{Yamaji2014}.
To utilize the parallel computing infrastructure with narrow bandwidth and distributed-memory
architectures, efficient, user-friendly, and highly parallelized diagonalization packages are highly desirable.

$\HPhi$, a flexible diagonalization package for solving quantum lattice hamiltonians,
has been developed to be such a descendant of the pioneering package TITPACK.
The Lanczos method for calculations of the ground state and a few excited states properties,
and finite temperature calculations based on thermal pure quantum states~\cite{Sugiura2012} are implemented in
the package $\HPhi$, with an easy-to-use and flexible user interface.
By using $\HPhi$, you can analyze a wide range of quantum lattice hamiltonians including
simple Hubbard and Heisenberg models, multi-band extensions of the Hubbard model,
exchange couplings that break SU(2) symmetry of quantum spins such as Dzyaloshinskii-Moriya
and Kitaev interactions, and Kondo lattice models describing itinerant electrons coupled with
quantum spins. $\HPhi$ calculates a variety of physical quantities such as internal energy at zero temperature or finite temperatures, temperature dependence of specific heat, charge/spin structure factors, and so on. A broad spectrum of users including experimental scientists is cordially welcome.

\subsection{License}
The distribution of the program package and the source codes for ${\mathcal H \Phi}$ follow GNU General Public License version 3 (GPL v3). 
\subsection{Copyright}
%{\it \copyright 2015- The University of Tokyo} {\it  All rights reserved.}
{\it \copyright 2015- T. Misawa, K. Yoshimi, M. Kawamura, Y. Yamaji, S. Todo and N. Kawashima.} {\it  All rights reserved.}\\
This software is developed under the support of ``{\it Project for advancement of software usability in materials science }" by The Institute for Solid State Physics, The University of Tokyo. 
%The copyright of this software belongs to The University of Tokyo.
\subsection{Contributors}
\label{subsec:contributors}
This software is developed by following contributors.
\begin{itemize}
\item{ver.0.2 (released at 2015/12/28)}
\item{ver.0.1 (released at 2015/10/09)}
\begin{itemize}
	\item{\bf Developers}
	\begin{itemize}
	\item{Takahiro Misawa \\(Department of Applied Physics, The University of Tokyo)}
	\item{Kazuyoshi Yoshimi\\ (The Institute for Solid State Physics, The University of Tokyo)}
	\item{Mitsuaki Kawamura\\ (The Institute for Solid State Physics, The University of Tokyo)}
	\end{itemize}
	\item{\bf Advisers}
	\begin{itemize}
	\item{Youhei Yamaji\\ (Department of Applied Physics, The University of Tokyo)}
	\item{Synge Todo\\ (Department of Physics, The University of Tokyo)}
	\end{itemize}
	\item{\bf Project coordinator}
	\begin{itemize}
	\item{Naoki Kawashima\\ (The Institute for Solid State Physics, The University of Tokyo)}
	\end{itemize}
\end{itemize}
\end{itemize}

\section{Operating environment}
 $\HPhi$ is tested in the following platforms:

\begin{itemize}
\item The supercomputer system-B ``sekirei'' and system-C ``maki'' in ISSP
\item Linux PC + intel compiler
\item Linux PC + gcc
\item Mac + gcc
\end{itemize}
