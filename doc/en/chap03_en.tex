% !TEX root = userguide_en.tex
%----------------------------------------------------------
\chapter{Tutorial}
\label{Ch:model}
\section{{\it Standard} mode}
% !TEX root = userguide_en.tex
\subsection{Heisenberg model}

This tutorial should be performed in 
\begin{verbatim}
samples/Standard/Spin/HeisenbergChain/
\end{verbatim}

The input file, reference outputs, and the redirected standard output are
provided as follows:
\begin{verbatim}
samples/Standard/Spin/HeisenbergChain/StdFace.def
samples/Standard/Spin/HeisenbergChain/output_FullDiag/
samples/Standard/Spin/HeisenbergChain/output_Lanczos/
samples/Standard/Spin/HeisenbergChain/output_TPQ/
samples/Standard/Spin/HeisenbergChain/FullDiag.out
samples/Standard/Spin/HeisenbergChain/Lanczos.out
samples/Standard/Spin/HeisenbergChain/TPQ.out
\end{verbatim}
%
In this case, we treat the one dimensional antiferromagnetic Heisenberg chain
which has a nearest neighbor spin coupling.
\begin{align}
  {\hat H} = J \sum_{i=1}^{N_{\rm site}} {\hat {\boldsymbol S}}_i \cdot {\hat {\boldsymbol S}}_{i+1}
\end{align}

The input file is as follows:
\\
\begin{minipage}{10cm}
\begin{screen}
\begin{verbatim}
L = 16
model = "Spin"
method = "Lanczos"
lattice = "chain lattice"
J = 1.0
2Sz = 0
\end{verbatim}
\end{screen}
\end{minipage}
%
\\
In this tutorial, J and the number of sites are set to 1 (arbitral unit)
and 16 respectively.

\subsubsection{Log output}
{Log messages are outputted to the standard output.
Log files for calculation procedure are made in "output" directory which is automatically created.
In this example, the following files are outputted.\\}
\begin{minipage}{16cm}
\begin{screen}
\begin{verbatim}
CHECK_InterAll.dat     Time_CG_EigenVector.dat  zvo_Lanczos_Step.dat  
CHECK_Memory.dat       WarningOnTransfer.dat    zvo_sz_TimeKeeper.dat
CHECK_Sdim.dat         zvo_TimeKeeper.dat
\end{verbatim}
\end{screen}
\end{minipage}
\\
The details for the outputted files are shown in \ref{Subsec:checkchemi}-\ref{Subsec:timecgeigenv}.
\\
We execute

\hspace{-0.7cm}
\verb|$ |\textit{Path}\verb|/HPhi -s StdFace.def|
\\
and obtain following standard outputs {(a compilation mode is MPI parallel/ Hybrid parallel)}:

\small
\begin{verbatim}
#####  Parallelization Info.  #####

  OpenMP threads : 4
  MPI PEs : 1 


######  Standard Intarface Mode STARTS  ######

  Open Standard-Mode Inputfile ./StdFace.def 

  KEYWORD : l                    | VALUE : 16 
  KEYWORD : model                | VALUE : spin 
  KEYWORD : method               | VALUE : lanczos 
  Skipping a line.
  Skipping a line.
  KEYWORD : lattice              | VALUE : chainlattice 
  KEYWORD : j                    | VALUE : 1.0 
  KEYWORD : 2sz                  | VALUE : 0 

#######  Parameter Summary  #######

                L = 16 
                a = 1.00000     ######  DEFAULT VALUE IS USED  ######
               2S = 1           ######  DEFAULT VALUE IS USED  ######
                h = 0.00000     ######  DEFAULT VALUE IS USED  ######
            Gamma = 0.00000     ######  DEFAULT VALUE IS USED  ######
                D = 0.00000     ######  DEFAULT VALUE IS USED  ######
                J = 1.00000   
               Jz = 1.00000     ######  DEFAULT VALUE IS USED  ######
              Jxy = 1.00000     ######  DEFAULT VALUE IS USED  ######
               Jx = 1.00000     ######  DEFAULT VALUE IS USED  ######
               Jy = 1.00000     ######  DEFAULT VALUE IS USED  ######
               J' = 0.00000     ######  DEFAULT VALUE IS USED  ######
              Jz' = 0.00000     ######  DEFAULT VALUE IS USED  ######
             Jxy' = 0.00000     ######  DEFAULT VALUE IS USED  ######
              Jx' = 0.00000     ######  DEFAULT VALUE IS USED  ######
              Jy' = 0.00000     ######  DEFAULT VALUE IS USED  ######
       LargeValue = 1           ######  DEFAULT VALUE IS USED  ######

         filehead = zvo         ######  DEFAULT VALUE IS USED  ######
      Lanczos_max = 2000        ######  DEFAULT VALUE IS USED  ######
       initial_iv = 1           ######  DEFAULT VALUE IS USED  ######
             nvec = 1           ######  DEFAULT VALUE IS USED  ######
             exct = 1           ######  DEFAULT VALUE IS USED  ######
       LanczosEps = 14          ######  DEFAULT VALUE IS USED  ######
    LanczosTarget = 2           ######  DEFAULT VALUE IS USED  ######
           NumAve = 5           ######  DEFAULT VALUE IS USED  ######
    ExpecInterval = 20          ######  DEFAULT VALUE IS USED  ######
              2Sz = 0  
      ioutputmode = 1           ######  DEFAULT VALUE IS USED  ######

######  Print Expert input files  ######

    zlocspin.def is written.
      zTrans.def is written.
   zInterAll.def is written.
    namelist.def is written.
     calcmod.def is written.
     modpara.def is written.
    greenone.def is written.
    greentwo.def is written.

######  Input files are generated.  ######

  Read File 'namelist.def'.
  Read File 'calcmod.def' for CalcMod.
  Read File 'modpara.def' for ModPara.
  Read File 'zlocspn.def' for LocSpin.
  Read File 'zTrans.def' for Trans.
  Read File 'zInterAll.def' for InterAll.
  Read File 'greenone.def' for OneBodyG.
  Read File 'greentwo.def' for TwoBodyG.

######  Definition files are correct.  ######

  Read File 'zlocspn.def'.
  Read File 'zTrans.def'.
  Read File 'zInterAll.def'.
  Read File 'greenone.def'.
  Read File 'greentwo.def'.

######  Indices and Parameters of Definition files(*.def) are complete.  ######

  MAX DIMENSION idim_max=12870 
  APPROXIMATE REQUIRED MEMORY  max_mem=0.001236 GB 


######  MPI site separation summary  ######

  INTRA process site
    Site    Bit
       0       2
       1       2
       2       2
       3       2
       4       2
       5       2
       6       2
       7       2
       8       2
       9       2
      10       2
      11       2
      12       2
      13       2
      14       2
      15       2

  INTER process site
    Site    Bit

  Process element info
    Process       Dimension   Nup  Ndown  Nelec  Total2Sz   State
          0           12870     8      8      8         0   

   Total dimension : 12870


######  LARGE ALLOCATE FINISH !  ######

  Start: Calculate HilbertNum for fixed Sz. 
  End  : Calculate HilbertNum for fixed Sz. 

  Start: Calculate diagaonal components of Hamiltonian. 
  End  : Calculate diagaonal components of Hamiltonian. 

######  Start: Calculate Lanczos Eigenvalue.  ######

  initial_mode=0 normal: iv = 6437 i_max=12870 k_exct =1 

  LanczosStep  E[1] E[2] E[3] E[4], E_Max / Nsite
  stp = 2 0.7192235936 2.7807764064 xxxxxxxxxx xxxxxxxxx 
  stp = 4 -1.1878294823 0.6833997592 2.1864630296 3.4242789858 0.2140174366 
  stp = 6 -2.4623912732 -0.8925857144 0.5104359160 1.7443963706 0.2277205490 
  stp = 8 -3.5878334190 -2.0969913075 -0.8250723080 0.3369317092 0.2325450330 
\end{verbatim}
\normalsize
( snip )
\small
\begin{verbatim}
  stp = 60 -7.1422963606 -6.8721066784 -6.6965474265 -6.5234070570 0.2499999989 
  stp = 62 -7.1422963606 -6.8721066784 -6.6965474266 -6.5234070573 0.2499999996 
  stp = 64 -7.1422963606 -6.8721066784 -6.6965474266 -6.5234070574 0.2499999999 

######  End  : Calculate Lanczos EigenValue.  ######

######  Start: Calculate Eigenvector.  ######

  Start: Calculate Lanczos Eigenvector.
  End  : Calculate Lanczos Eigenvector.
  Start: Calculate Energy.
  End  : Calculate Energy.

  Accuracy check !!!
  LanczosEnergy = -7.14229636061676e+00 
  EnergyByVec   = -7.14229636061616e+00 
  diff_ene      = 8.50586433832080e-14 
  var           = 9.97303843188625e-14 
  Accuracy of Lanczos vectors is enough.

######  End  : Calculate Eigenvector.  ######

  Start: Calculate one body Green functions.
  End  : Calculate one body Green functions.

  Start: Calculate two bodies Green functions.
  End  : Calculate two bodies Green functions.
\end{verbatim}
\normalsize

In the beginning of this run,
files describing the detail of considered Hamiltonian 
(\verb|zlocspin.def|, \verb|zTrans.def|, \verb|zInterAll.def|, 
\verb|namelist.def|, \verb|calcmod.def|, \verb|modpara.def|)
and files specifying elements of correlation functions
that will be calculated
(\verb|greenone.def|, \verb|greentwo.def|)
are generated.

\begin{screen}
\Large 
{\bf Tips}
\normalsize

In the standard outputs at each even step, 
the quantity of \verb|E_Max / Nsite| column indicates 
the maximum energy per sites
(In this example, it converges to \verb|0.2499999999=0.25|).
In the TPQ calculation, you can set \verb|Largevalue| %to an integer 
larger than \verb|E_Max / Nsite|.

\end{screen}

\subsubsection{Outputs for calculation results}
\begin{description}
\item {\bf Lanczos method}\\
When a calculation by Lanczos method is finished normally, eigenenergies, one-body Green's functions and two-body Green's functions are calculated and outputted to the files, respectively. In this sample, following files are outputted.\\

\begin{minipage}{12cm}
\begin{screen}
\begin{verbatim}
zvo_energy.dat zvo_cisajs.dat 
zvo_cisajscktalt.dat  
\end{verbatim}
\end{screen}
\end{minipage}

For standard mode, all pairs of $\langle n_{i\sigma} \rangle$ are calculated as one-body Green's functions and those of $\langle n_{i\sigma} n_{j\sigma'} \rangle$ are calculated as two-body Green's functions on the basis of the definition files, \verb|greenone.def| and \verb|greentwo.def|. \\
When accuracy of Lanczos vectors is enough, one-body and two-body Green's functions are calculated by eigenvectors obtained by Lanczos method. While accuracy of Lanczos vectors is {\it not} enough, a message ``Accuracy of Lanczos vector is not enough" is outputted to the standard output and one-body and two-body Green's functions are calculated by eigenvectors obtained by CG method.
The details of output files are shown in \ref{subsec:energy.dat}, \ref{Subsec:cgcisajs}, \ref{Subsec:cisajscktalt}.

\item {\bf TPQ method}\\
When \verb|method="TPQ"| is selected in an input file, a calculation by TPQ method is started. After finishing of the calculation normally, following files are outputted, where \%\% is a number of run and \&\& is a number of steps for TPQ method.\\
\begin{minipage}{14cm}
\begin{screen}
\begin{verbatim}
Norm_rand%%.dat SS_rand%%.dat
zvo_cisajs_set%%step&&.dat  
zvo_cisajscktalt_set%%step&&.dat  
\end{verbatim}
\end{screen}
\end{minipage}

In Norm\_rand\%\%.dat, basic informations such as inverse of temperature and a norm of wave function before normalization are outputted with a TPQ step for each number of runs. In SS\_rand\%\%.dat, physical quantities such as inverse of temperature, energy and expected value of square of Hamiltonian are outputted with a TPQ step for each number of runs.  In zvo\_cisajs\_set\%\%step\&\&.dat and zvo\_cisajscktalt\_set\%\%step\&\&.dat, one-body and two-body Green's functions are outputted for each number of a TPQ steps and runs.
The details of these files are shown in \ref{Subsec:normrand}, \ref{Subsec:ssrand}, \ref{Subsec:cgcisajs}, \ref{Subsec:cisajscktalt}.

\item {\bf Full diagonalization method}\\
When \verb|method = "fulldiag"| is selected in an input file, a calculation by Full diagonalization method is started.
After finishing of the calculation normally, following files are outputted, where xx is a number of an eigenstate counting from 0. \\
\begin{minipage}{14cm}
\begin{screen}
\begin{verbatim}
Eigenvalue.dat zvo_cisajs_eigen_xx.dat
zvo_cisajscktalt_eigen_xx.dat  zvo_phys_Nup4_Ndown4.dat
\end{verbatim}
\end{screen}
\end{minipage}

In Eigenvalue.dat, an eigennumber and an eigenvalue are outputted for each lines.
In zvo\_cisajs\_eigen\_xx.dat and zvo\_cisajscktalt\_eigen\_xx.dat, one-body Green's functions and two-body Green's functions are outputted for each eigennumber. 
In zvo\_phys\_Nup4\_Ndown4.dat, physical quantities such as expected values of energy and doublon are outputted. The details of these files are shown in \ref{Subsec:eigenvalue} - \ref{Subsec:cisajscktalt}.


\end{description}

\subsection{Other tutorials}

There are following tutorials in
\verb|samples/Standard/|.

\begin{itemize}
\item The Hubbard model on the two dimensional square lattice

  (\verb|samples/Standard/Hubbard/square/|)
\item The Hubbard model on the two dimensional triangular lattice
  
  (\verb|samples/Standard/Hubbard/triangular/|)
\item The one dimensional Kondo chain

  (\verb|samples/Standard/Kondo/chain/|)
\item The one dimensional antiferromagnetic Heisenberg chain
  
  (\verb|samples/Standard/Spin/HeisenbergChain/HeisenbergChain/|)
\item The antiferromagnetic Heisenberg model on the two dimensional square lattice
  
  (\verb|samples/Standard/Spin/HeisenbergSquare/|)
\item The Kitaev model with 2$\times$3 unit cells to Honeycomb lattice
  
  (\verb|samples/Standard/Spin/Kitaev/|)
\end{itemize}

We can perform these tutorials in the same way to the previous one.


%----------------------------------------------------------
\section{{\it Expert} mode}
For expert mode, following input files are needed
\begin{enumerate}
\item A file list for input files,
\item Files for basic parameters,
\item Files for constructing Hamiltonian,
\item Files for setting output components.
\end{enumerate}
The process after calculation is same as standard mode. In this section, we show the demonstration for one dimensional antiferromagnetic Heisenberg chain model which has a nearest neighbor spin coupling, 
\begin{equation}
H=\sum_{i=0}^{15} J {\bm S}_i\cdot {\bm S}_{i+1},
\end{equation}
where $J=2$, ${\bm S}_{16}={\bm S}_{0}$.\\
We use following input files in samples/Expert/Spin/HeisenbergChain.\\
\begin{minipage}{15cm}
\begin{screen}
\begin{verbatim}
calcmod.def   greentwo.def  namelist.def  zTrans.def
greenone.def  modpara.def   zInterAll.def zlocspn.def
\end{verbatim}
\end{screen}
\end{minipage}

\subsection{A file list for input files}
In namelist.def, kinds of input files and filenames are defined as shown below.
By writing keyword and filenames at each lines, kinds of files are distinguished. The details for namelist.def are shown in  \ref{Subsec:InputFileList}.
\\
\begin{minipage}{15cm}
\begin{screen}
\begin{verbatim}
CalcMod calcmod.def
ModPara modpara.def
LocSpin zlocspn.def
Trans zTrans.def
InterAll zInterAll.def
OneBodyG greenone.def
TwoBodyG greentwo.def
\end{verbatim}
\end{screen}
\end{minipage}

\subsection{Files for basic parameters}
In this subsection, we show the way to set a calculation mode, parameters for calculation and positions of localized spins.
\begin{description}
\item {\bf Setting a calculation mode}\\
Calculation mode is set in a CalcMod file (in this sample file, calcmod.def). The contents of files are shown as below.\\
\begin{minipage}{15cm}
\begin{screen}
\begin{verbatim}
#CalcType = 0:Lanczos, 1:TPQCalc, 2:FullDiag
#CalcMod = 0:Hubbard, 1:Spin, 2:Kondo, 3:HubbardGC, 
4:SpinGC, 5:KondoGC 
CalcType   0
CalcModel   1
\end{verbatim}
\end{screen}
\end{minipage}
~\\
We select a calculation method by CalcType and a target model by CalcModel. In this sample, we set Lanczos method as a calculation method and a target model as a spin system (canonical ensemble). The details of a CalcMod file are shown in \ref{Subsec:calcmod}.\\

\item {\bf Setting parameters for calculation}\\
Parameter for calculation are set in a ModPara file (in this sample, modpara.def). The contents of files are shown as below.\\
\begin{minipage}{15cm}
\begin{screen}
\begin{verbatim}
--------------------
Model_Parameters   0
--------------------
VMC_Cal_Parameters
--------------------
CDataFileHead  zvo
CParaFileHead  zqp
--------------------
Nsite          16   
Ncond          0    
2Sz            0    
Lanczos_max    1000 
initial_iv     12   
nvec           1    
exct           1    
LanczosEps     14   
LanczosTarget  2    
LargeValue     12   
NumAve         5    
ExpecInterval  20 
\end{verbatim}
\end{screen}
\end{minipage}
~\\
In this file, we set parameters for calculation such as a site number, {a total number of conduction electrons,  a total $S_z$} and a number of Lanczos step etc. The details of ModPara file are shown in \ref{Subsec:modpara}.\\

\item {\bf Setting positions of localized spins}\\
Positions {and $S$} of localized spins are defined by a LocSpin file (in this sample, locspn.def). The contents of files are shown as below.\\
\begin{minipage}{15cm}
\begin{screen}
\begin{verbatim}
================================ 
NlocalSpin    16  
================================ 
========i_0LocSpn_1IteElc ====== 
================================ 
    0      1
    1      1
    2      1
    3      1
    4      1
    5      1
…
\end{verbatim}
\end{screen}
\end{minipage}
~\\
When CalcModel in a CalcMod file is set as spin system, all sites automatically treated as localized spins. The details of a LosSpn file is shown in \ref{Subsec:locspn}.
\end{description}

\subsection{Files for constructing Hamiltonian}
After setting basic parameters, we make input files for constructing Hamiltonian. Since calculations are done by the representation of fermion operators in $\HPhi$, we must rewrite spin operator. {For example,  in the case of $S=1/2$, we rewrite the equation by} using the following relation
\begin{align}
S_z^{(i)}&=(c_{i\uparrow}^{\dag}c_{i\uparrow}-c_{i\downarrow}^{\dag}c_{i\downarrow})/2,\\
S_+^{(i)}&=c_{i\uparrow}^{\dag}c_{i\downarrow},\\
S_-^{(i)}&=c_{i\downarrow}^{\dag}c_{i\uparrow}.
\end{align}

\begin{description}
\item {\bf Setting transfer integrals}\\
In a Trans file (in this sample, zTrans.def), we set a transfer part of Hamiltonian,
\begin{align}
H +=-\sum_{ij\sigma_1\sigma2} t_{ij\sigma_1\sigma2}c_{i\sigma_1}^{\dag}c_{j\sigma_2}.
\end{align}
The contents of files are shown as below.\\
\begin{minipage}{15cm}
\begin{screen}
\begin{verbatim}
======================== 
NTransfer       0  
======================== 
========i_j_s_tijs====== 
======================== 
\end{verbatim}
\end{screen}
\end{minipage}
~\\
We can use this term when an electric magnetic field is added in spin system. For example, when an magnetic field is added at a site 1 such as $-0.5 S_z^{(1)}$ { for $S=1/2$}, this term can be rewritten as $-0.5/2(c_{1\uparrow}^{\dag}c_{1\uparrow}-c_{1\downarrow}^{\dag}c_{1\downarrow})$. Thus, the input file becomes as follows.\\
\begin{minipage}{15cm}
\begin{screen}
\begin{verbatim}
======================== 
NTransfer      1   
======================== 
========i_j_s_tijs====== 
======================== 
1 0 1 0 -0.25 0
1 1 1 1 0.25 0
\end{verbatim}
\end{screen}
\end{minipage}
~\\
The details for a Trans file are shown in \ref{Subsec:Trans}.

\item {\bf Setting general two-body interactions}\\
In an InterAll file (in this sample, zInterall.def), we set a general two-body interaction part of Hamiltonian,
\begin{equation}
H+=\sum_{i,j,k,l}\sum_{\sigma_1,\sigma_2, \sigma_3, \sigma_4}
I_{ijkl\sigma_1\sigma_2\sigma_3\sigma_4}c_{i\sigma_1}^{\dagger}c_{j\sigma_2}c_{k\sigma_3}^{\dagger}c_{l\sigma_4}.
\end{equation}
The contents of files are shown as below.\\
\begin{minipage}{16cm}
\begin{screen}
\begin{verbatim}
====================== 
NInterAll      96  
====================== 
========zInterAll===== 
====================== 
    0     0     0     0     1     0     1     0   0.500000  0.000000
    0     0     0     0     1     1     1     1  -0.500000  0.000000
    0     1     0     1     1     0     1     0  -0.500000  0.000000
    0     1     0     1     1     1     1     1   0.500000  0.000000
    0     0     0     1     1     1     1     0   1.000000  0.000000
    0     1     0     0     1     0     1     1   1.000000  0.000000
…
\end{verbatim}
\end{screen}
\end{minipage}
~\\
Here, we explain an interaction between a site $i$ and a site $j$ {in the case of $S=1/2$}, for simplicity. Using fermion operators, interaction terms for spin operators can be rewritten as
\begin{align}
H_{i,i+1}&=J(S_x^{(i)}S_x^{(i+1)}+S_y^{(i)}S_y^{(i+1)}+S_z^{(i)}S_z^{(i+1)}) \nonumber\\
&=J \left( \frac{1}{2}S_+^{(i)}S_-^{(i+1)}+\frac{1}{2}S_-^{(i)}S_+^{(i+1)}+S_z^{(i)}S_z^{(i+1)} \right) \nonumber\\
&=J \left[ \frac{1}{2}c_{i\uparrow}^{\dag}c_{i\downarrow}c_{i+1\downarrow}^{\dag}c_{i+1\uparrow}+\frac{1}{2}c_{i\downarrow}^{\dag}c_{i\uparrow}c_{i+1\uparrow}^{\dag}c_{i+1\downarrow}+\frac{1}{4}(c_{i\uparrow}^{\dag}c_{i\uparrow}-c_{i\downarrow}^{\dag}c_{i\downarrow})(c_{i+1\uparrow}^{\dag}c_{i+1\uparrow}-c_{i+1\downarrow}^{\dag}c_{i+1\downarrow}) \right]. \nonumber 
\end{align}
Thus, the interaction $S_z^{(i)}S_z^{(i+1)}$ for $J=2$ can be written as \\
\begin{minipage}{16cm}
\begin{screen}
\begin{verbatim}
    i     0     i     0    i+1     0    i+1     0   0.500000  0.000000
    i     0     i     0    i+1     1    i+1     1  -0.500000  0.000000
    i     1     i     1    i+1     0    i+1     0  -0.500000  0.000000
    i     1     i     1    i+1     1    i+1     1   0.500000  0.000000
\end{verbatim}
\end{screen}
\end{minipage}
~\\
in the format of an InterAll file. The other terms can be written as below.\\
\begin{minipage}{16cm}
\begin{screen}
\begin{verbatim}
    i     0     i     1    i+1     1    i+1     0   1.000000  0.000000
    i     1     i     0    i+1     0    i+1     1   1.000000  0.000000
\end{verbatim}
\end{screen}
\end{minipage}
~\\
There are other file formats for constructing Hamiltonian. The details for input formats of two-body interactions are shown in \ref{Subsec:interall}-\ref{Subsec:pairlift}.

\end{description}

\subsection{Setting output components}
In OneBodyG and TwoBodyG files, calculating components for one-body and two-body Green's functions are defined, respectively. 
\begin{description}
\item {\bf Setting calculating components for one-body Green's functions}\\
In a OneBodyG file (in this sample, greenone.def), calculating components for $\langle c_{i\sigma_1}^{\dag}c_{j\sigma_2} \rangle$ are defined. The contents of files are shown as below.\\
\begin{minipage}{15cm}
\begin{screen}
\begin{verbatim}
===============================
NCisAjs         32
===============================
======== Green functions ======
===============================
    0     0     0     0
    0     1     0     1
    1     0     1     0
    1     1     1     1
    2     0     2     0
…
\end{verbatim}
\end{screen}
\end{minipage}
~\\
The details for input formats of a OneBodyG file are shown in \ref{Subsec:onebodyg}.
\item {\bf Setting calculating components for two-body Green's functions}\\
In a TwoBodyG file (in this sample, greentwo.def), calculating components for $\langle c_{i\sigma_1}^{\dag}c_{j\sigma_2}c_{k\sigma_3}^{\dag}c_{l\sigma_4} \rangle$ are defined. The contents of files are shown as below.\\
\begin{minipage}{15cm}
\begin{screen}
\begin{verbatim}
=============================================
NCisAjsCktAltDC       1024
=============================================
======== Green functions for Sq AND Nq ======
=============================================
    0     0     0     0     0     0     0     0
    0     0     0     0     0     1     0     1
    0     0     0     0     1     0     1     0
    0     0     0     0     1     1     1     1
    0     0     0     0     2     0     2     0
    …
\end{verbatim}
\end{screen}
\end{minipage}
~\\
The details for input formats of a TwoBodyG file are shown in \ref{Subsec:twobodyg}.
\end{description}

\subsection{Running}
After making all input files above, we ready to run a program. For expert mode, we must set an option ``-e" and a file name list (in this sample, namelist.def) as arguments to run $\HPhi$.\\
\verb|$ |\textit{Path}\verb|/HPhi -e namelist.def|\\
The process after calculating is same as that of standard mode.

\subsection{Other tutorials}

There are following tutorials in \verb|samples/Expert/|.

\begin{itemize}
\item The Hubbard model on the two dimensional square lattice

  (\verb|samples/Expert/Hubbard/square/|)
\item The Hubbard model on the two dimensional triangular lattice
  
  (\verb|samples/Expert/Hubbard/triangular/|)
\item The one dimensional Kondo chain

  (\verb|samples/Expert/Kondo/chain/|)
\item The one dimensional antiferromagnetic Heisenberg chain
  
  (\verb|samples/Expert/Spin/HeisenbergChain/HeisenbergChain/|)
\item The antiferromagnetic Heisenberg model on the two dimensional square lattice
  
  (\verb|samples/Expert/Spin/HeisenbergSquare/|)
      
\item The Kitaev model with 2$\times$3 unit cells to Honeycomb lattice

  (\verb|samples/Expert/Spin/Kitaev/|)

\end{itemize}

We can perform these tutorials in the same way to the previous one.

\newpage
\section{Making input files for {\it Expert} mode}
This mode is to make input files for {\it Expert} mode. A set of input files made by this mode gives a model provided in {\it Standard} mode.
The usage is shown as below.
\begin{enumerate}
\item{Make an input file for {\it Standard} mode.}
\item{Setting an option ``-sdry" and an input file (in this example, StdFace.def), run $\HPhi$.}\\
\verb|$ |\textit{Path}\verb|/HPhi -sdry StdFace.def|
\item{Following files are made as input files for {\it Expert} mode in the present working directory.}\\
\begin{minipage}{12cm}
\begin{screen}
\begin{verbatim}
calcmod.def   greentwo.def  namelist.def  zTrans.def
greenone.def  modpara.def   zInterAll.def zlocspn.def
\end{verbatim}
\end{screen}
\end{minipage}
\end{enumerate}
