% !TEX root = userguide_en.tex
\subsection{Heisenberg model}

This tutorial should be performed in 
\begin{verbatim}
samples/Standard/Spin/HeisenbergChain/
\end{verbatim}

The input file, reference outputs, and the redirected standard output are
provided as follows:
\begin{verbatim}
samples/Standard/Spin/HeisenbergChain/StdFace.def
samples/Standard/Spin/HeisenbergChain/output_FullDiag/
samples/Standard/Spin/HeisenbergChain/output_Lanczos/
samples/Standard/Spin/HeisenbergChain/output_TPQ/
samples/Standard/Spin/HeisenbergChain/FullDiag.out
samples/Standard/Spin/HeisenbergChain/Lanczos.out
samples/Standard/Spin/HeisenbergChain/TPQ.out
\end{verbatim}
%
In this case, we treat the one dimensional antiferromagnetic Heisenberg chain
which has a nearest neighbor spin coupling.
\begin{align}
  {\hat H} = J \sum_{i=1}^{N_{\rm site}} {\hat {\boldsymbol S}}_i \cdot {\hat {\boldsymbol S}}_{i+1}
\end{align}

The input file is as follows:
\\
\begin{minipage}{10cm}
\begin{screen}
\begin{verbatim}
L = 16
model = "Spin"
method = "Lanczos"
lattice = "chain lattice"
J = 1.0
2Sz = 0
\end{verbatim}
\end{screen}
\end{minipage}
%
\\
In this tutorial, J and the number of sites are set to 1 (arbitral unit)
and 16 respectively.

\subsubsection{Log output}
{Log messages are outputted to the standard output.
Log files for calculation procedure are made in "output" directory which is automatically created.
In this example, the following files are outputted.\\}
\begin{minipage}{16cm}
\begin{screen}
\begin{verbatim}
CHECK_InterAll.dat     Time_CG_EigenVector.dat  zvo_Lanczos_Step.dat  
CHECK_Memory.dat       WarningOnTransfer.dat    zvo_sz_TimeKeeper.dat
CHECK_Sdim.dat         zvo_TimeKeeper.dat
\end{verbatim}
\end{screen}
\end{minipage}
\\
The details for the outputted files are shown in \ref{Subsec:checkchemi}-\ref{Subsec:timecgeigenv}.
\\
We execute

\hspace{-0.7cm}
\verb|$ |\textit{Path}\verb|/HPhi -s StdFace.def|
\\
and obtain following standard outputs \tr{(a compilation mode is MPI parallel/ Hybrid parallel)}:
\begin{verbatim}
#####  Parallelization Info.  #####

  OpenMP threads : 4
  MPI PEs : 1 


######  Standard Intarface Mode STARTS  ######

  Open Standard-Mode Inputfile ./StdFace.def 

  KEYWORD : l                    | VALUE : 16 
  KEYWORD : model                | VALUE : spin 
  KEYWORD : method               | VALUE : lanczos 
  Skipping a line.
  Skipping a line.
  KEYWORD : lattice              | VALUE : chainlattice 
  KEYWORD : j                    | VALUE : 1.0 
  KEYWORD : 2sz                  | VALUE : 0 

#######  Parameter Summary  #######

                L = 16 
                a = 1.00000     ######  DEFAULT VALUE IS USED  ######
               2S = 1           ######  DEFAULT VALUE IS USED  ######
                h = 0.00000     ######  DEFAULT VALUE IS USED  ######
            Gamma = 0.00000     ######  DEFAULT VALUE IS USED  ######
                D = 0.00000     ######  DEFAULT VALUE IS USED  ######
                J = 1.00000   
               Jz = 1.00000     ######  DEFAULT VALUE IS USED  ######
              Jxy = 1.00000     ######  DEFAULT VALUE IS USED  ######
               Jx = 1.00000     ######  DEFAULT VALUE IS USED  ######
               Jy = 1.00000     ######  DEFAULT VALUE IS USED  ######
               J' = 0.00000     ######  DEFAULT VALUE IS USED  ######
              Jz' = 0.00000     ######  DEFAULT VALUE IS USED  ######
             Jxy' = 0.00000     ######  DEFAULT VALUE IS USED  ######
              Jx' = 0.00000     ######  DEFAULT VALUE IS USED  ######
              Jy' = 0.00000     ######  DEFAULT VALUE IS USED  ######
       LargeValue = 1           ######  DEFAULT VALUE IS USED  ######

         filehead = zvo         ######  DEFAULT VALUE IS USED  ######
      Lanczos_max = 2000        ######  DEFAULT VALUE IS USED  ######
       initial_iv = 1           ######  DEFAULT VALUE IS USED  ######
             nvec = 1           ######  DEFAULT VALUE IS USED  ######
             exct = 1           ######  DEFAULT VALUE IS USED  ######
       LanczosEps = 14          ######  DEFAULT VALUE IS USED  ######
    LanczosTarget = 2           ######  DEFAULT VALUE IS USED  ######
           NumAve = 5           ######  DEFAULT VALUE IS USED  ######
    ExpecInterval = 20          ######  DEFAULT VALUE IS USED  ######
              2Sz = 0  
      ioutputmode = 1           ######  DEFAULT VALUE IS USED  ######

######  Print Expert input files  ######

    zlocspin.def is written.
      zTrans.def is written.
   zInterAll.def is written.
    namelist.def is written.
     calcmod.def is written.
     modpara.def is written.
    greenone.def is written.
    greentwo.def is written.

######  Input files are generated.  ######

  Read File 'namelist.def'.
  Read File 'calcmod.def' for CalcMod.
  Read File 'modpara.def' for ModPara.
  Read File 'zlocspn.def' for LocSpin.
  Read File 'zTrans.def' for Trans.
  Read File 'zInterAll.def' for InterAll.
  Read File 'greenone.def' for OneBodyG.
  Read File 'greentwo.def' for TwoBodyG.

######  Definition files are correct.  ######

  Read File 'zlocspn.def'.
  Read File 'zTrans.def'.
  Read File 'zInterAll.def'.
  Read File 'greenone.def'.
  Read File 'greentwo.def'.

######  Indices and Parameters of Definition files(*.def) are complete.  ######

  MAX DIMENSION idim_max=12870 
  APPROXIMATE REQUIRED MEMORY  max_mem=0.001236 GB 


######  MPI site separation summary  ######

  INTRA process site
    Site    Bit
       0       2
       1       2
       2       2
       3       2
       4       2
       5       2
       6       2
       7       2
       8       2
       9       2
      10       2
      11       2
      12       2
      13       2
      14       2
      15       2

  INTER process site
    Site    Bit

  Process element info
    Process       Dimension   Nup  Ndown  Nelec  Total2Sz   State
          0           12870     8      8      8         0   

   Total dimension : 12870


######  LARGE ALLOCATE FINISH !  ######

  Start: Calculate HilbertNum for fixed Sz. 
  End  : Calculate HilbertNum for fixed Sz. 

  Start: Calculate diagaonal components of Hamiltonian. 
  End  : Calculate diagaonal components of Hamiltonian. 

######  Start: Calculate Lanczos Eigenvalue.  ######

  initial_mode=0 normal: iv = 6437 i_max=12870 k_exct =1 

  LanczosStep  E[1] E[2] E[3] E[4] 
  stp = 2 0.7192235936 2.7807764064 xxxxxxxxxx xxxxxxxxx 
  stp = 4 -1.1878294823 0.6833997592 2.1864630296 3.4242789858 
  stp = 6 -2.4623912732 -0.8925857144 0.5104359160 1.7443963706 
  stp = 8 -3.5878334190 -2.0969913075 -0.8250723080 0.3369317092 
\end{verbatim}
(中略)
\begin{verbatim}
  stp = 60 -7.1422963606 -6.8721066784 -6.6965474265 -6.5234070570 
  stp = 62 -7.1422963606 -6.8721066784 -6.6965474266 -6.5234070573 
  stp = 64 -7.1422963606 -6.8721066784 -6.6965474266 -6.5234070574 

######  End  : Calculate Lanczos EigenValue.  ######

######  Start: Calculate Eigenvector.  ######

  Start: Calculate Lanczos Eigenvector.
  End  : Calculate Lanczos Eigenvector.
  Start: Calculate Energy.
  End  : Calculate Energy.

  Accuracy check !!!
  LanczosEnergy = -7.14229636061676e+00 
  EnergyByVec   = -7.14229636061616e+00 
  diff_ene      = 8.50586433832080e-14 
  var           = 9.97303843188625e-14 
  Accuracy of Lanczos vectors is enough.

######  End  : Calculate Eigenvector.  ######

  Start: Calculate one body Green functions.
  End  : Calculate one body Green functions.

  Start: Calculate two bodies Green functions.
  End  : Calculate two bodies Green functions.\end{verbatim}

In the beginning of this run,
files describing the detail of considered Hamiltonian 
(\verb|zlocspin.def|、\verb|zTrans.def|、\verb|zInterAll.def|、
\verb|namelist.def|、\verb|calcmod.def|、\verb|modpara.def|)
and files specifying elements of correlation functions
that will be calculated
(\verb|greenone.def|、\verb|greentwo.def|)
are generated.

\subsubsection{Outputs for calculation results}
\begin{description}
\item {\bf Lanczos method}\\
When a calculation by Lanczos method is finished normally, eigenenergies, one-body Green's functions and two-body Green's functions are calculated and outputted to the files, respectively. In this sample, following files are outputted.\\

\begin{minipage}{12cm}
\begin{screen}
\begin{verbatim}
zvo_energy.dat zvo_cisajs.dat 
zvo_cisajscktalt.dat  
\end{verbatim}
\end{screen}
\end{minipage}

For standard mode, all pairs of $\langle n_{i\sigma} \rangle$ are calculated as one-body Green's functions and those of $\langle n_{i\sigma} n_{j\sigma'} \rangle$ are calculated as two-body Green's functions on the basis of the definition files, \verb|greenone.def| and \verb|greentwo.def|. \\
When accuracy of Lanczos vectors is enough, one-body and two-body Green's functions are calculated by eigenvectors obtained by Lanczos method. While accuracy of Lanczos vectors is {\it not} enough, a message ``Accuracy of Lanczos vector is not enough" is outputted to the standard output and one-body and two-body Green's functions are calculated by eigenvectors obtained by CG method.
The details of output files are shown in \ref{subsec:energy.dat}, \ref{Subsec:cgcisajs}, \ref{Subsec:cisajscktalt}.

\item {\bf TPQ method}\\
When \verb|method="TPQ"| is selected in an input file, a calculation by TPQ method is started. After finishing of the calculation normally, following files are outputted, where \%\% is a number of run and \&\& is a number of steps for TPQ method.\\
\begin{minipage}{14cm}
\begin{screen}
\begin{verbatim}
Norm_rand%%.dat SS_rand%%.dat
zvo_cisajs_set%%step&&.dat  
zvo_cisajscktalt_set%%step&&.dat  
\end{verbatim}
\end{screen}
\end{minipage}

In Norm\_rand\%\%.dat, basic informations such as inverse of temperature and a norm of wave function before normalization are outputted with a TPQ step for each number of runs. In SS\_rand\%\%.dat, physical quantities such as inverse of temperature, energy and expected value of square of Hamiltonian are outputted with a TPQ step for each number of runs.  In zvo\_cisajs\_set\%\%step\&\&.dat and zvo\_cisajscktalt\_set\%\%step\&\&.dat, one-body and two-body Green's functions are outputted for each number of a TPQ steps and runs.
The details of these files are shown in \ref{Subsec:normrand}, \ref{Subsec:ssrand}, \ref{Subsec:cgcisajs}, \ref{Subsec:cisajscktalt}.

\item {\bf Full diagonalization method}\\
When \verb|method = "fulldiag"| is selected in an input file, a calculation by Full diagonalization method is started.
After finishing of the calculation normally, following files are outputted, where xx is a number of an eigenstate counting from 0. \\
\begin{minipage}{14cm}
\begin{screen}
\begin{verbatim}
Eigenvalue.dat zvo_cisajs_eigen_xx.dat
zvo_cisajscktalt_eigen_xx.dat  zvo_phys_Nup4_Ndown4.dat
\end{verbatim}
\end{screen}
\end{minipage}

In Eigenvalue.dat, an eigennumber and an eigenvalue are outputted for each lines.
In zvo\_cisajs\_eigen\_xx.dat and zvo\_cisajscktalt\_eigen\_xx.dat, one-body Green's functions and two-body Green's functions are outputted for each eigennumber. 
In zvo\_phys\_Nup4\_Ndown4.dat, physical quantities such as expected values of energy and doublon are outputted. The details of these files are shown in \ref{Subsec:eigenvalue} - \ref{Subsec:cisajscktalt}.


\end{description}

\subsection{Other tutorials}

There are following tutorials in
\verb|samples/Standard/|.

\begin{itemize}
\item The Hubbard model on the two dimensional square lattice

  (\verb|samples/Standard/Hubbard/square/|)
\item The Hubbard model on the two dimensional triangular lattice
  
  (\verb|samples/Standard/Hubbard/triangular/|)
\item The one dimensional Kondo chain

  (\verb|samples/Standard/Kondo/chain/|)
\item The one dimensional antiferromagnetic Heisenberg chain
  
  (\verb|samples/Standard/Spin/HeisenbergChain/HeisenbergChain/|)
\item The antiferromagnetic Heisenberg model on the two dimensional square lattice
  
  (\verb|samples/Standard/Spin/HeisenbergSquare/|)
\item The Kitaev model with 2$\times$3 unit cells to Honeycomb lattice
  
  (\verb|samples/Standard/Spin/Kitaev/|)
\end{itemize}

We can perform these tutorials in the same way to the previous one.
