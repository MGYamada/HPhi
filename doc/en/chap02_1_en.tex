% !TEX root = userguide_en.tex
%----------------------------------------------------------
You can download $\HPhi$ in the following place.\\
\url{https://github.com/QLMS/HPhi/releases}

You can obtain the $\HPhi$ directory by tiping
\begin{verbatim}
$ tar xzvf HPhi-xxx.tar.gz
\end{verbatim}

There are two kind of procedures to install $\HPhi$.

\subsection{Using \texttt{HPhiconfig.sh}}

Please run \verb|HPhiconfig.sh| script in the $\HPhi$ directory as follow
(for ISSP system-B ''sekirei''):
\begin{verbatim}
$ bash HPhiconfig.sh sekirei
\end{verbatim}
Then environmental configuration file \verb|make.sys| is generated in 
\verb|src/| directory.
The command-line argment of \verb|HPhiconfig.sh| is as folows:
\begin{itemize}
\item \verb|sekirei| : ISSP system-B ''sekirei''
\item \verb|maki| : ISSP system-C ''maki''
\item \verb|intel| : intel compiler + Linux PC
\item \verb|mpicc-intel| : intel compiler + MPI (excepting intelMPI) + Linux PC
\item \verb|gcc| : GCC + Linux PC
\item \verb|gcc-mac| : GCC + Mac
\end{itemize}

\verb|make.sys| is as follows (for ISSP-system-B ''sekirei''):
\begin{verbatim}
 CC = mpiicc
 LAPACK_FLAGS = -Dlapack -mkl=parallel 
 FLAGS = -qopenmp  -O3 -xCORE-AVX2 -mcmodel=large -shared-intel -D MPI
 MTFLAGS = -DDSFMT_MEXP=19937 $(FLAGS)
 INCLUDE_DIR=./include
\end{verbatim}
We explatin macros of this file as: 
\begin{itemize}
\item \verb|CC| : The Compilation command (\verb|icc|, \verb|gcc|, \verb|fccpx|)
\item \verb|LAPACK_FLAGS| : Compilation options for LAPACK. \verb|-Dlapack| can not be removed.
\item \verb|FLAGS| : Other compilation options.
  OpenMP utilization option (\verb|-openmp|, \verb|-fopenmp|, \verb|-qopenmp|, etc.)
  must be specified.
  When you use MPI, please set \verb|-D MPI|.
\item \verb|MTFLAGS|, \verb|INCLUDE_DIR| : Options for the Mersenne Twister
  and additional include directory. You do not have to modify them.
\end{itemize}


Then you are ready to compile HPhi.
Please type
\begin{verbatim}
$ make HPhi
\end{verbatim}
and obtain an executable \verb|HPhi| in \verb|src/| directory;
you should add this directory to the \verb|$PATH|.



\begin{screen}
\Large 
{\bf Tips}
\normalsize

You can make a PATH to $\HPhi$ as follows:
\\
\verb|$ export PATH=${PATH}:|\textit{HPhi\_top\_directory}\verb|/src/|
\\
If you keep this PATH, you should write above in \verb|~/.bashrc|
(for \verb|bash| as a login shell)

\end{screen}


\subsection{Using \texttt{cmake}}
We can compile Hphi as
\begin{verbatim}
cd $HOME/build/hphi
cmake -DCONFIG=gcc $PathTohphi
make
\end{verbatim}
Here, we set a path to $\HPhi$ as \verb| $PathTohphi| and to a build directory as \verb| $HOME/build/hphi|. After compiling, a src folder is constructed below a \verb| $HOME/build/hphi |folder and obtain an executable \verb|HPhi| in \verb|src/| directory. When there is not a MPI library in your system, an executable \verb|HPhi| is automatically compiled without a MPI library.

In the above example, we compile $\HPhi$ by using a gcc compiler. We can select a compiler by using following options
\begin{itemize}
\item \verb|sekirei| : ISSP system-B ''sekirei''
\item \verb|fujitsu| : Fujitsu compiler (ISSP system-C ''maki'')
\item \verb|intel| : intel compiler + Linux PC
\item \verb|gcc| : GCC compiler + Linux PC.
\end{itemize}
An example for compiling $\HPhi$ by an intel compiler is shown as follows, 
\begin{verbatim}
mkdir ./build
cd ./build
cmake -DCONFIG=intel ../
make
\end{verbatim}
After compiling,  a \verb|src| folder is made below the \verb|build| folder and an execute $\HPhi$ is made in the  \verb|src| folder. It is noted that  we must delete the  \verb|build| folder and do the above works again when we change the compilers.
