% !TEX root = userguide_en.tex
%----------------------------------------------------------
\begin{enumerate}

\item You can download $\HPhi$ in the following place.\\
\url{https://github.com/QLMS/HPhi/releases}

\item After expanding downloaded files, 
move into \verb|src| in the expanded directory.
\begin{verbatim}
$ tar xzvf 
$ cd /src/
\end{verbatim}

\item Configure the environment for the installation;
  please edit \verb|make.sys| and specify the compiler and the option.
  For some known system, 
  you can find prepared settings as comments in \verb|make.sys| and use them.
  For example, when you install $\HPhi$ into the ISSP System-B(sekirei),
  find 
\begin{verbatim}
#####  ISSP System-B "sekirei" #####|

# CC = icc
# LAPACK_FLAGS = -Dlapack -mkl=parallel 
# FLAGS = -qopenmp  -O3 -xCORE-AVX2 -mcmodel=large -shared-intel
# MTFLAGS = -DDSFMT_MEXP=19937 $(FLAGS)
# INCLUDE_DIR=./include
\end{verbatim}
and uncomment it.

\item Save the modified \verb|make.sys| and type \verb|make| as follows:
\begin{verbatim}
$ make all
\end{verbatim}

Then an executable \verb|HPhi| in the current directory;
you should add this directory to the \verb|$PATH| or
make a symbolic link in a place belonging \verb|$PATH|.
\begin{verbatim}
$ sudo ln -fs `pwd`/HPhi /usr/local/bin/
\end{verbatim}

\begin{screen}
\Large 
{\bf Tips}
\normalsize

You can make a PATH to $\HPhi$ as follows:
\\
\verb|$ export PATH=${PATH}:|\textit{HPhi\_top\_directory}\verb|/src/|
\\
If you keep this PATH, you should write above in \verb|~/.bashrc|
(for \verb|bash| as a login shell)

\end{screen}

\end{enumerate}
