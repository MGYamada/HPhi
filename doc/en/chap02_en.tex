% !TEX root = userguide_en.tex
%----------------------------------------------------------
\chapter{How to use $\HPhi$?}
\label{Ch:HowTo}

\section{Prerequisite}

$\HPhi$ requires the following packages:
\begin{itemize}
\item C compiler (intel, Fujitsu, GNU, etc. )
\item LAPACK library (intel MKL, Fujitsu, ATLAS, etc.)
\item MPI library (If you do not use MPI, it is not necessary.)
\end{itemize}

\begin{screen}
\Large 
{\bf Tips}
\normalsize

{\bf E. g. / Settings of intel compiler}

When you use the intel compiler, you can use easily scripts attached to the compiler.
In the case of the bash in 64 bit OS, write the following in your \verb|~/.bashrc|:
\begin{verbatim}
source /opt/intel/bin/compilervars.sh intel64
\end{verbatim}
or
\begin{verbatim}
source /opt/intel/bin/iccvars.sh intel64
source /opt/intel/mkl/bin/mklvars.sh
\end{verbatim}

Please read manuals of your compiler/library for more information.

\end{screen}

\section{Installation}

% !TEX root = userguide_en.tex
%----------------------------------------------------------
You can download $\HPhi$ in the following place.\\
\url{https://github.com/QLMS/HPhi/releases}

You can obtain the $\HPhi$ directory by typing
\begin{verbatim}
$ tar xzvf HPhi-xxx.tar.gz
\end{verbatim}

There are two kind of procedures to install $\HPhi$.

\subsection{Using \texttt{HPhiconfig.sh}}

Please run \verb|HPhiconfig.sh| script in the $\HPhi$ directory as follow
(for ISSP system-B ''sekirei''):
\begin{verbatim}
$ bash HPhiconfig.sh sekirei
\end{verbatim}
Then environmental configuration file \verb|make.sys| is generated in 
\verb|src/| directory.
The command-line argument of \verb|HPhiconfig.sh| is as follows:
\begin{itemize}
\item \verb|sekirei| : ISSP system-B ''sekirei''
\item \verb|maki| : ISSP system-C ''maki''
\item \verb|sr| : HITACHI SR16000
\item \verb|intel| : intel compiler + Linux PC
\item \verb|mpicc-intel| : intel compiler + MPI (excepting intelMPI) + Linux PC
\item \verb|gcc| : GCC + Linux PC
\item \verb|gcc-mac| : GCC + Mac
\end{itemize}

\verb|make.sys| is as follows (for ISSP-system-B ''sekirei''):
\begin{verbatim}
 CC = mpiicc
 LAPACK_FLAGS = -Dlapack -mkl=parallel 
 FLAGS = -qopenmp  -O3 -xCORE-AVX2 -mcmodel=large -shared-intel -D MPI
 MTFLAGS = -DDSFMT_MEXP=19937 $(FLAGS)
 INCLUDE_DIR=./include
\end{verbatim}
We explain macros of this file as: 
\begin{itemize}
\item \verb|CC| : The Compilation command (\verb|icc|, \verb|gcc|, \verb|fccpx|)
\item \verb|LAPACK_FLAGS| : Compilation options for LAPACK. \verb|-Dlapack| can not be removed.
\item \verb|FLAGS| : Other compilation options.
  OpenMP utilization option (\verb|-openmp|, \verb|-fopenmp|, \verb|-qopenmp|, etc.)
  must be specified.
  When you use MPI, please set \verb|-D MPI|.
\item \verb|MTFLAGS|, \verb|INCLUDE_DIR| : Options for the Mersenne Twister
  and additional include directory. You do not have to modify them.
\end{itemize}


Then you are ready to compile HPhi.
Please type
\begin{verbatim}
$ make HPhi
\end{verbatim}
and obtain an executable \verb|HPhi| in \verb|src/| directory;
you should add this directory to the \verb|$PATH|.



\begin{screen}
\Large 
{\bf Tips}
\normalsize

You can make a PATH to $\HPhi$ as follows:
\\
\verb|$ export PATH=${PATH}:|\textit{HPhi\_top\_directory}\verb|/src/|
\\
If you keep this PATH, you should write above in \verb|~/.bashrc|
(for \verb|bash| as a login shell)

\end{screen}


\subsection{Using \texttt{cmake}}

\begin{screen}
\Large 
{\bf Tips}
\normalsize\\
Before using cmake for sekirei, you must type 
\begin{verbatim}
source /home/issp/materiapps/tool/env.sh
\end{verbatim}
while for maki, you must type
\begin{verbatim}
source /global/app/materiapps/tool/env.sh
\end{verbatim}
\end{screen}

We can compile Hphi as
\begin{verbatim}
cd $HOME/build/hphi
cmake -DCONFIG=gcc $PathTohphi
make
\end{verbatim}
Here, we set a path to $\HPhi$ as \verb| $PathTohphi| and to a build directory as \verb| $HOME/build/hphi|. After compiling, a src folder is constructed below a \verb| $HOME/build/hphi |folder and obtain an executable \verb|HPhi| in \verb|src/| directory. When there is not a MPI library in your system, an executable \verb|HPhi| is automatically compiled without a MPI library.

In the above example, we compile $\HPhi$ by using a gcc compiler. We can select a compiler by using following options
\begin{itemize}
\item \verb|sekirei| : ISSP system-B ''sekirei''
\item \verb|fujitsu| : Fujitsu compiler (ISSP system-C ''maki'')
\item \verb|intel| : intel compiler + Linux PC
\item \verb|gcc| : GCC compiler + Linux PC.
\end{itemize}
An example for compiling $\HPhi$ by an intel compiler is shown as follows, 
\begin{verbatim}
mkdir ./build
cd ./build
cmake -DCONFIG=intel ../
make
\end{verbatim}
After compiling,  a \verb|src| folder is made below the \verb|build| folder and an execute $\HPhi$ is made in the  \verb|src| folder. It is noted that  we must delete the  \verb|build| folder and do the above works again when we change the compilers.

\label{Sec:HowToInstall}

\section{Directory structure}
When HPhi-xxx.tar.gz is unzipped, the following directory structure is composed.
 
\begin{verbatim}
|--CMakeLists.txt
|--COPYING
|--config/
|    |--fujitsu.cmake
|    |--gcc.cmake
|    |--intel.cmake
|    ---sekirei.cmake
|--doc/
|    |--en/
|    |--jp/
|    |--userguide_en.pdf
|    ---userguide_jp.pdf
|--HPhiconfig.sh
|--samples/
|    |--Expert/
|    |    |--Hubbard/
|    |    |    |-square/
|    |    |    |      |--*.def
|    |    |    |      |--output_FullDiag/*.dat
|    |    |    |      |--output_Lanczos/*.dat
|    |    |    |      ---output_TPQ/*.dat
|    |    |    --triangular/...
|    |    |--Kondo/chain/...
|    |    ---Spin/
|    |         |-HeisenbergChain/...
|    |         |-HeisenbergSquare/...
|    |         |-Kagome/...
|    |         |-Kagome36Boost/...
|    |         |-Kitaev/...
|    |         --S1Chain/...
|    ---Standard/
|         |--Hubbard/
|         |    |-square/
|         |    |   |--StdFace.def
|         |    |   |--output_FullDiag/*.dat
|         |    |   |--output_Lanczos/*.dat
|         |    |   ---output_TPQ/*.dat
|         |    --triangular/...
|         |--Kondo/chain/...
|         ---Spin/
|              |-HeisenbergChain/...
|              |-HeisenbergSquare/...
|              |-Kitaev/...
|              --S1Chain/...
|--src/
|    |--*.c
|    |--CMakeLists.txt
|    |--include/*.h
|    |--makefile_src
|    ---StdFace/
|--test/
---test_tool/
\end{verbatim}

\section{Basic usage}
$\HPhi$ has two modes; standard mode and expert mode. Here, the basic flows of calculations by standard and expert modes are shown.

\subsection{{\it Standard} mode}

The procedure of calculation through the standard mode is shown as follows:

\begin{enumerate}

\item  Make a directory for a calculation scenario. 

First, you make a working directory for the calculation.

\item  Make input files for standard mode

In the standard mode, you can choose a model (the Heisenberg model, the Hubbard model, etc.) and 
a lattice (the square lattice, the triangular lattice, etc.) from ones provided;
you can specify some parameters (such as the first/second nearest neighbor hopping integrals,
the on-site Coulomb integral, etc.) for them.
Finally, you have to specify the numerical method (such as the Lanczos method) employed in this calculation.
The input file format is described in the Sec. \ref{Ch:HowToStandard}.

\item  Run

Run a executable \verb|HPhi| in terminal by setting option ``\verb|-s|" (or ``\verb|--standard|'') and the name of input file written in previous step.

\begin{itemize}

\item Serial/OpenMP parallel

  \verb|$ |\textit{Path}\verb|/HPhi -s |\textit{Input\_file\_name}

\item MPI parallel/ Hybrid parallel

  \verb|$ mpiexec -np |\textit{number\_of\_processes}\verb| |\textit{Path}\verb|/HPhi -s |\textit{Input\_file\_name}

  When you use a queuing system in workstations or super computers, 
  sometimes the number of processes is specified as an argument for the job-submitting command.
  If you need more information, please refer manuals for your system. 
 A number of processes depend on a target of system for models. The details of setting a number of processes are shown in  \ref{subsec:process}.
\end{itemize}

\item Watch calculation logs

Log files are outputted in the ``output" folder which is automatically made in the directory for a calculation scenario.
The details of output files are shown in \ref{Sec:outputfile}.

\item Results

If the calculation is finished normally, the result files are outputted in  the ``output" folder. The details of output files are shown in \ref{Sec:outputfile}.

\end{enumerate}

\begin{screen}
\Large 
{\bf Tips}
\normalsize

{\bf The number of threads for OpenMP}

If you specify the number of OpenMP threads for $\HPhi$,
you should set it as follows (in case of 16 threads) before the running:
\begin{verbatim}
export OMP_NUM_THREADS=16
\end{verbatim}

\end{screen}

\subsection{{\it Expert} mode}
The procedure of calculation for expert mode is shown as follows.
 \begin{enumerate}
   \item  Make a directory for a calculation scenario. \\
First, you make a directory named as a calculation scenario (you can attach an arbitrary name to a directory).
   \item  Make input files for expert mode\\
For expert mode,  you should make input files 
for constructing Hamiltonian operators, calculation condition and 
a list file for the filenames of input files (see the file formats shown in  \ref{Ch:HowToExpert}). \\
{\bf Note:} A List file can be easily made by using standard mode.
 \item  Run\\
Run $\HPhi$ in terminal by setting option ``\verb|-e|" (or ``\verb|--expert|'') and a file name of a list file.\\

\begin{itemize}

\item Serial/OpenMP

  \verb|$ |\textit{Path}\verb|/HPhi -e |\textit{Input\_List\_file\_name}
  
\item MPI/Hybrid

  \verb|$ mpiexec -np |\textit{number\_of\_processes}\verb| |\textit{Path}\verb|/HPhi -e |\textit{Input\_List\_file\_name}
 {A number of processes depend on a target of system for models. The details of setting a number of processes are shown in  \ref{subsec:process}.}
\end{itemize}

\item Under running\\
Log files are outputted in the ``output" folder which is automatically made in the directory for a calculation scenario.
The details of output files are shown in \ref{Sec:outputfile}.

\item Results\\
If the calculation is finished normally, the result files are outputted in  the ``output" folder. The details of output files are shown in \ref{Sec:outputfile}.
\end{enumerate}



\newpage
\subsection{Setting a process number for MPI/Hybrid parallelization}
\label{subsec:process}
{For using MPI/Hybrid parallelization, a process number must be set as follows.}
\begin{enumerate}
\item{Standard mode}

\begin{itemize}
\item{Hubbard / Kondo model}\\
When \verb|model| in an input file for Standard mode is set as \verb|"Fermion Hubbard"|, \verb|"Kondo Lattice"| or \verb|"Fermion HubbardGC"|,  a process number must be equal to $4^n$.

\item{Spin model}\\
When \verb|model| in an input file for Standard mode is set as \verb|"Spin"| or \verb|"SpinGC"|,  a process number must be equal to (\verb|2S|+1)${}^n$ where \verb|2S| is set in the input file (a default value is $1$).

\end{itemize}
\item{Expert mode}

\begin{itemize}
\item{Hubbard / Kondo model}

When a model is selected as fermion Hubbard model or Kondo model by setting \verb|CalcModel| in a {\bf CalcMod} file, a process number must be equal to $4^n$.
See \ref{Subsec:calcmod} for details of a \verb|CalcModel| file.

\item{Spin model}

When a model is selected as Spin model by setting \verb|CalcModel| in a {\bf CalcMod} file, a process number is fixed by a {\bf LocSpin} file. A process number must be equal to a number calculated by multiplying a state number of localized spin (\verb|2S|+1) in descending order about site numbers.
See \ref{Subsec:locspn} for details of a {\bf LocSpin} file.

For example, when a {\bf LocSpin} file is given as follws, a process number must be equal to $2=1+1,~6=2\times(2+1),~24=6\times(3+1)$.
\begin{minipage}{10cm}
\begin{screen}
\begin{verbatim}
================================ 
NlocalSpin     3
================================ 
========i_0IteElc_2S ====== 
================================ 
    0      3
    1      2
    2      1
\end{verbatim}
\end{screen}
\end{minipage}

\end{itemize}

\subsection{Printing version ID}

By using \verb|-v| option as follows, 
you can check which version of $\HPhi$ you are using.

\begin{verbatim}
$ PATH/HPhi -v
\end{verbatim}

\end{enumerate}

