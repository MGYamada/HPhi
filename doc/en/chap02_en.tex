% !TEX root = userguide_en.tex
%----------------------------------------------------------
\chapter{How to use $\HPhi$}
\label{Ch:HowTo}

\section{Prerequisite}

$\HPhi$ requires the following packages:
\begin{itemize}
\item C/fortran compiler (Intel, Fujitsu, GNU, etc. )
\item BLAS/LAPACK library (Intel MKL, Fujitsu, ATLAS, etc.)
\item MPI library (if you do not use MPI, this is not requied).
\end{itemize}

\begin{screen}
\Large 
{\bf Tips}
\normalsize

{\bf E.g. /Settings of Intel compiler}

When you use the Intel compiler, you can easily use the scripts attached to the compiler.
In the case of the bash in the 64-bit OS, write the following in your \verb|~/.bashrc|:
\begin{verbatim}
source /opt/intel/bin/compilervars.sh intel64
\end{verbatim}
or
\begin{verbatim}
source /opt/intel/bin/iccvars.sh intel64
source /opt/intel/mkl/bin/mklvars.sh
\end{verbatim}

Please read the manuals of your compiler/library for more information.

\end{screen}

\section{Installation}

% !TEX root = userguide_en.tex
%----------------------------------------------------------
You can download $\HPhi$ at the following location.\\
\url{https://github.com/QLMS/HPhi/releases}

You can obtain the $\HPhi$ directory by typing
\begin{verbatim}
$ tar xzvf HPhi-xxx.tar.gz
\end{verbatim}

There are two types of procedure for installing $\HPhi$.

\subsection{Using \texttt{HPhiconfig.sh}}

Please run \verb|HPhiconfig.sh| script in the $\HPhi$ directory as follows
(for ISSP system-B ``sekirei''):
\begin{verbatim}
$ bash HPhiconfig.sh sekirei
\end{verbatim}
Then, the environmental configuration file \verb|make.sys| is generated in 
the \verb|src/| directory.
The command-line argument of \verb|HPhiconfig.sh| is as follows:
\begin{itemize}
\item \verb|sekirei| : ISSP system-B ``sekirei''
\item \verb|fujitsu| : ISSP system-C ``maki''
\item \verb|sr| : HITACHI SR16000
\item \verb|intel| : Intel compiler
\item \verb|intel-openmpi| : Intel compiler + OpenMPI
\item \verb|intel-mpich| : Intel compiler + MPICH2
\item \verb|intel-intelmpi| : Intel compiler + IntelMPI
\item \verb|gcc| : GCC
\item \verb|gcc-openmpi| : GCC + OpenMPI
\item \verb|gcc-mpich| : GCC + MPICH2
\item \verb|gcc-mac| : GCC + Mac.
\end{itemize}

\verb|make.sys| is as follows (for ISSP-system-B ``sekirei''):
\begin{verbatim}
CC = mpicc
F90 = mpif90
CFLAGS = -fopenmp -O3 -g -traceback -xHost -ipo -mcmodel=large \
         -shared-intel -D MPI
FFLAGS = -fopenmp -O3 -g -traceback -xHost -ipo -mcmodel=large \
         -shared-intel -D MPI -fpp
LIBS = -mkl -lifcore
\end{verbatim}
We explain the macros of this file as: 
\begin{itemize}
\item \verb|CC| : C compiler (\verb|icc|, \verb|gcc|, \verb|fccpx|).
\item \verb|F90| : fortran compiler (\verb|ifort|, \verb|gfortran|, \verb|frtpx|)
\item \verb|CFLAGS| : C compile options.
  OpenMP utilization option (\verb|-openmp|, \verb|-fopenmp|, \verb|-qopenmp|, etc.)
  must be specified.
  When you use MPI, please set \verb|-D MPI|.
\item \verb|FFLAGS| : fortran compile options. Similar to \verb|CFLAGS|. 
\item \verb|LIBS| : Compilation options for LAPACK. \verb|-Dlapack| can not be removed.
\end{itemize}


Then, you are ready to compile HPhi.
Please type
\begin{verbatim}
$ make HPhi
\end{verbatim}
and obtain an executable \verb|HPhi| in the \verb|src/| directory;
you should add this directory to the \verb|$PATH|.



\begin{screen}
\Large 
{\bf Tips}
\normalsize

You can make a PATH to $\HPhi$ as follows:
\\
\verb|$ export PATH=${PATH}:|\textit{HPhi\_top\_directory}\verb|/src/|
\\
If you retain this PATH, you should write above in \verb|~/.bashrc|
(for \verb|bash| as a login shell)

\end{screen}


\subsection{Using \texttt{cmake}}

\begin{screen}
\Large 
{\bf Tips}
\normalsize\\
Before using cmake for sekirei, you must type 
\begin{verbatim}
source /home/issp/materiapps/tool/env.sh
\end{verbatim}
while for maki, you must type
\begin{verbatim}
source /global/app/materiapps/tool/env.sh
\end{verbatim}
\end{screen}

We can compile $\HPhi$ as
\begin{verbatim}
cd $HOME/build/hphi
cmake -DCONFIG=gcc $PathTohphi
make
\end{verbatim}
Here, we set a path to $\HPhi$ as \verb|$PathTohphi|
and to a build directory as \verb| $HOME/build/hphi|.
After compilation, \verb|src| folder is constructed below a \verb|$HOME/build/hphi|
folder and we obtain an executable \verb|HPhi| in \verb|src/| directory.
When no MPI library exists in the system, an executable \verb|HPhi|
is automatically compiled without an MPI library.

In the above example,
we compile $\HPhi$ by using a gcc compiler.
We can select a compiler by using the following options:
\begin{itemize}
\item \verb|sekirei| : ISSP system-B ``sekirei''
\item \verb|fujitsu| : Fujitsu compiler (ISSP system-C ``maki'')
\item \verb|intel| : Intel compiler + Linux PC
\item \verb|gcc| : GCC compiler + Linux PC.
\end{itemize}
An example of compiling $\HPhi$ by using the Intel compiler is shown as follows:
\begin{verbatim}
mkdir ./build
cd ./build
cmake -DCONFIG=intel ../
make
\end{verbatim}
After compilation,
\verb|src| folder is created below the \verb|build| folder and
an execute $\HPhi$ in the \verb|src| folder.
Please note that we must delete the \verb|build| folder and
repeat the above operations when we change the compiler.

\label{Sec:HowToInstall}

\section{Directory structure}
When HPhi-xxx.tar.gz is unzipped, the following directory structure is composed.
 
\begin{verbatim}
|--CMakeLists.txt
|--COPYING
|--config/
|    |--fujitsu.cmake
|    |--gcc.cmake
|    |--intel.cmake
|    ---sekirei.cmake
|--doc/
|    |--en/
|    |--jp/
|    |--fourier/
|    |    |--en/
|    |    |--figs/
|    |    ---ja/
|    |--userguide_en.pdf
|    ---userguide_jp.pdf
|--HPhiconfig.sh
|--samples/
|    |--Expert/
|    |    |--Hubbard/
|    |    |    |-square/
|    |    |    |      |--*.def
|    |    |    |      |--output_FullDiag/*.dat
|    |    |    |      |--output_Lanczos/*.dat
|    |    |    |      ---output_TPQ/*.dat
|    |    |    --triangular/...
|    |    |--Kondo/chain/...
|    |    ---Spin/
|    |         |-HeisenbergChain/...
|    |         |-HeisenbergSquare/...
|    |         |-Kagome/...
|    |         |-Kagome36Boost/...
|    |         |-Kitaev/...
|    |         --S1Chain/...
|    ---Standard/
|         |--Hubbard/
|         |    |-square/
|         |    |   |--StdFace.def
|         |    |   |--output_FullDiag/*.dat
|         |    |   |--output_Lanczos/*.dat
|         |    |   ---output_TPQ/*.dat
|         |    --triangular/...
|         |--Kondo/chain/...
|         ---Spin/
|              |-HeisenbergChain/...
|              |-HeisenbergSquare/...
|              |-Kitaev/...
|              --S1Chain/...
|--src/
|    |--*.c
|    |--CMakeLists.txt
|    |--include/*.h
|    |--makefile_src
|    ---StdFace/
|--test/
---test_tool/
\end{verbatim}

\section{Basic usage}
$\HPhi$ has two modes: Standard mode and Expert mode.
Here, the basic flows of calculations of the Standard and expert modes are shown.

\subsection{{\it Standard} mode}

The procedure of calculation through the standard mode is as follows:

\begin{enumerate}

\item Create a directory for a calculation scenario

First, you create a working directory for the calculation.

\item Create input files for Standard mode

In Standard mode, you can choose a model (the Heisenberg model, Hubbard model, etc.) and 
a lattice (the square lattice, triangular lattice, etc.) from those provided;
you can specify some parameters
(such as the first/second nearest neighbor hopping integrals and
the on-site Coulomb integral) for them.
Finally, you have to specify the numerical method (such as the Lanczos method) employed in this calculation.
The input file format is described in Sec. \ref{Ch:HowToStandard}.

\item  Run

  Run an executable \verb|HPhi| in the terminal by setting option ``\verb|-s|"
  (or ``\verb|--standard|'') and the name of the input file written in the previous step.

\begin{itemize}

\item Serial/OpenMP parallel

  \verb|$ |\textit{Path}\verb|/HPhi -s |\textit{Input\_file\_name}

\item MPI parallel/ Hybrid parallel

  \verb|$ mpiexec -np |\textit{number\_of\_processes}\verb| |\textit{Path}\verb|/HPhi -s |\textit{Input\_file\_name}

  When you use a queuing system in workstations or super computers, 
  sometimes the number of processes is specified as an argument for the job-submitting command.
  If you need more information, please refer to your system manuals. 
  The number of processes depends on the target system of the models.
  The details of setting the number of processes are shown in Sec. \ref{subsec:process}.
\end{itemize}

\item Watch calculation logs

  Log files are outputted in the ``output'' folder,
  which is automatically created in the directory for a calculation scenario.
  The details of the output files are shown in Sec. \ref{Sec:outputfile}.

\item Results

  If the calculation is completed normally,
  the result files are outputted in  the ``output'' folder.
  The details of the output files are shown in Sec. \ref{Sec:outputfile}.

\end{enumerate}

\begin{screen}
\Large 
{\bf Tips}
\normalsize

{\bf The number of threads for OpenMP}

If you specify the number of OpenMP threads for $\HPhi$,
you should set it as follows (in the case of 16 threads) before running:
\begin{verbatim}
export OMP_NUM_THREADS=16
\end{verbatim}

\end{screen}

\subsection{{\it Expert} mode}
The calculation procedure for Expert mode is as follows.
 \begin{enumerate}
   \item Create a directory for a calculation scenario \\
     First, you create a directory and give it the name of a calculation scenario
     (you can attach an arbitrary name to a directory).
   \item Create input files for Expert mode\\
     For Expert mode, you should create input files 
     for constructing Hamiltonian operators, calculation conditions, and 
     a list file for the filenames of the input files
     (see the file formats shown in Sec. \ref{Ch:HowToExpert}). \\
     {\bf Note:} A list file can be easily created by using Standard mode.
 \item  Run\\
   Run $\HPhi$ in the terminal by setting option ``\verb|-e|''
   (or ``\verb|--expert|'') and the file name of a list file.\\

\begin{itemize}

\item Serial/OpenMP

  \verb|$ |\textit{Path}\verb|/HPhi -e |\textit{Input\_List\_file\_name}
  
\item MPI/Hybrid

  \verb|$ mpiexec -np |\textit{number\_of\_processes}\verb| |\textit{Path}\verb|/HPhi -e |\textit{Input\_List\_file\_name}
 {A number of processes depend on a target of system for models. The details of setting a number of processes are shown in  \ref{subsec:process}.}
\end{itemize}

\item While running\\
  Log files are outputted in the ``output" folder which is automatically created
  in the directory for a calculation scenario.
  The details of the output files are shown in Sec. \ref{Sec:outputfile}.

\item Results\\
  If the calculation is finished normally,
  the result files are outputted in the ``output'' folder.
  The details of the output files are shown in Sec. \ref{Sec:outputfile}.
\end{enumerate}



\newpage
\subsection{Setting the process number for MPI/hybrid parallelization}
\label{subsec:process}
For using MPI/hybrid parallelization, the process number must be set as follows.
\begin{enumerate}
\item{Standard mode}

\begin{itemize}
\item{Hubbard/Kondo model}\\
  When \verb|model| in the input file for Standard mode is set as
  \verb|"Fermion Hubbard"|, \verb|"Kondo Lattice"|, or \verb|"Fermion HubbardGC"|,
  the process number must be equal to $4^n$.

\item{Spin model}\\
  When \verb|model| in the input file for Standard mode is set as
  \verb|"Spin"| or \verb|"SpinGC"|, the process number must be equal to
  (\verb|2S|+1)${}^n$, where \verb|2S| is set in the input file
  (the default value is $1$).

\end{itemize}
\item{Expert mode}

\begin{itemize}
\item{Hubbard/Kondo model}

  When the model is selected as the Fermion Hubbard model or Kondo model
  by setting \verb|CalcModel| in a {\bf CalcMod} file,
  the process number must be equal to $4^n$.
  See Sec. \ref{Subsec:calcmod} for details of the \verb|CalcModel| file.

\item{Spin model}

  When the model is selected as the spin model by setting \verb|CalcModel|
  in a {\bf CalcMod} file, the process number is fixed by a {\bf LocSpin} file.
  The process number must be equal to the number calculated by
  multiplying the state number of the localized spin (\verb|2S|+1)
  in descending order by the site number.
  See Sec. \ref{Subsec:locspn} for details of the {\bf LocSpin} file.

  For example, when a {\bf LocSpin} file is given as follows,
  the process number must be equal to $2=1+1,~6=2\times(2+1),~24=6\times(3+1)$.
\begin{minipage}{10cm}
\begin{screen}
\begin{verbatim}
================================ 
NlocalSpin     3
================================ 
========i_0IteElc_2S ====== 
================================ 
    0      3
    1      2
    2      1
\end{verbatim}
\end{screen}
\end{minipage}

\end{itemize}

\subsection{Printing version ID}

By using the \verb|-v| option as follows, 
you can check which version of $\HPhi$ you are using.

\begin{verbatim}
$ PATH/HPhi -v
\end{verbatim}

\end{enumerate}

