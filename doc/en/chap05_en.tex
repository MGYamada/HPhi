% !TEX root = userguide_en.tex
%----------------------------------------------------------
\chapter{ Algorithm}
\label{Ch:algorithm}
\section{Lanczos method}
\label{Ch:Lanczos}
\subsection{Details of Lanczos method}
Some parts of this section are based on the manual
of titpack~\cite{titpack} and textbook by M. Sugihara and 
K. Murota~\cite{SugiharaMurota_en}(These references are written in Japanese).

In the Lanczos method, by successively operating the Hamiltonian 
to the initial vector, we obtain the accurate eigenvalues around
the maximum and minimum eigenvalues and associated eigenvectors.  
Because we can perform Lanczos method by using only two
vectors whose dimensions are the dimension of the total Hilbert space
~\footnote{In $\HPhi$, to reduce the numerical cost,
we use some additional vectors; vector for accumulating the 
real-space diagonal elements of the Hamiltonian, 
vector for specifying the given $S_{z}$ space and given particle space.
The dimension of these vectors is that of the Hilbert space.},
Lanczos method is frequently used for the 
diagonalization of the large matrices.
As we detail below,
one additional vector is necessary for
obtaining the eigenvector.

The principle of the Lanczos method is
based on the power method.
In the power method,
by successively operating the Hamiltonian $\Ham$ to the
arbitrary vector $\vec{x}_{0}$, we generate $\Ham^{n}\vec{x}_{0}$.
The obtained space 
$\mathcal{K}_{n+1}(\Ham,\vec{x}_{0})=\{\vec{x}_{0},\Ham^{1}\vec{x}_{0},\dots,\Ham^{n}\vec{x}_{0}\}$
is called Krylov subspace.
Initial vector is represented by the superposition 
of the eigenvectors 
$\vec{e}_{i}$ (corresponding eigenvalues are $E_{i}$) of $\Ham$ as 
\begin{align}
\vec{x}_{0}=\sum_{i}a_{i}\vec{e}_{i}.
\end{align}
Here, $E_{0}$ is maximum absolute values of the eigenvalues.
We note that all the eigenvalues are real number because Hamiltonian is Hermite.
By operating $\Ham^{n}$ to the initial vector,
we obtain the relation as
\begin{align}
\Ham^{n}\vec{x}_{0}=E_{0}^{n}\Big[ a_{0}\vec{e}_{0}+\sum_{i\neq0}\left(\frac{E_{i}}{E_{0}}\right)^na_{i}\vec{e}_{i}\Big].
\end{align}
This relation indicates that
the eigenvector of $E_{0}$ becomes dominant for sufficiently large $n$. 
In the Lanczos method,
we obtain the eigenvalues and eigenvectors 
by performing the proper transformation for obtained Krylov subspace.

In the Lanczos method,
we successively generate the normalized orthogonal basis 
${\vec{v}_{0},\dots,\vec{v}_{n-1}}$ from the Krylov subspace $\mathcal{K}_{n}(\Ham,\vec{x}_{0})$.
We defines initial vector and associated components as 
$\vec{v}_{0} =\vec{x}_{0}/|\vec{x}_{0}|$,
$\beta_{0}=0,\vec{x}_{-1}=0$.
From this initial condition,
we can obtain the normalized orthogonal basis  as follows:
\begin{align}
\alpha_{k} &= (\Ham\vec{v}_{k},\vec{v}_{k}), \\
\vec{w}   &= \Ham\vec{v}_{k}-\beta_{k}\vec{v}_{k-1}-\alpha_{k}\vec{v}_{k}, \\
\beta_{k+1} &= |\vec{w}|, \\
\vec{v}_{k+1} &= \frac{\vec{v}_{k}}{|\vec{v}_{k}|}.
\end{align}
From these definitions, it it obvious that $\alpha_{k}$, $\beta_{k}$ are real number.

In the subspace spanned by these normalized orthogonal basis,
the Hamiltonian is transformed as
\begin{align}
T_{n}=V_{n}^{\dagger}\Ham V_{n}.
\end{align}
Here,
$V_{n}$ is matrix whose column vectors are 
$\vec{v}_{i}(i=0,1,\dots,n-1)$.
$T_{n}$ is tridiagonal matrix and its diagonal elements
are $\alpha_{i}$ and
subdiagonal elements are $\beta_{i}$.
It is known that
the eigenvalues of $\Ham$ are well approximated by 
the eigenvalues of $T_{n}$ for sufficiently large $n$.
(We note that $V^{\dagger}V=I$,$I$ is identity matrix).
The original eigenvectors of $\Ham$ is obtained 
by $\vec{e}_{i}=V\tilde{\vec{e}}_{i}$,
where  $\tilde{\vec{e}}_{i}$ are
the eigenvectors of $T_{n}$.
From $V$, 
we can obtain the eigenvectors of $\Ham$
by performing the Lanczos method.
However, in the actual calculations,
it is difficult to keep $V$ because its dimension
is large [dimension of $V$ = (dimension of the total Hilbert space) $\times$ ($\#$ of Lanczos iterations)].
Thus, to obtain the eigenvectors, 
we again perform the same Lanczos calculations
after we obtain the eigenvalues from the Lanczos methods.
In the first Lanczos calculation, we keep $\tilde{\vec{e}_{i}}$
because its dimension is small
\footnote{upper bound of the dimensions of $\tilde{\vec{e}_{i}}$ is $\#$ of Lanczos iterations.}.
From this procedure, we obtain the eigenvectors  from $V$.

In the Lanczos method,
within a few hundred or thousand Lanczos iterations,
we obtain the accurate eigenvalues near the maximum and minimum values of eigenvalues.
The necessary number of iterations is small enough 
compared to the dimensions
of the total Hilbert space.
%Thus, the dimension of $T_{n}$ is 
%typically less than 100-1000.
We note that it is shown that
the errors of the maximum and minimum eigenvalues
becomes exponentially small as a function of Lanczos iteration 
(for details, see Ref.~\cite{SugiharaMurota_en}).

\subsection{Inverse iteration method}

From the approximate value of the eigenvalues ($E_{n}$),
by successively operating $(\Ham-E_{n})^{-1}$
to the initial vector $\vec{y}_{0}$,
we can obtain the accurate eigenvector for $E_{n}$.

From $(\Ham-E_{n})^{-1}\vec{y}_{0}$,
we obtain the linear simultaneous equations such as  
\begin{align}
\vec{y}_{k}&=(\Ham-E_{n})\vec{y}_{k+1}.
\end{align}
By solving this equation by using the
conjugate gradient method (CG method),
we obtain the eigenvector.
From the obtained eigenvector,
we can calculate the eigenvalues and correlation functions. 
We note that additional four vectors are necessary to
perform the CG method.
For large system size,
it may be impossible to allocate memory to the
additional vectors.

\subsection{Details of implementation}
In $\HPhi$,
we use \verb|dsyev| (routine of lapack)
for diagonalization of $T_{n}$.
We use the energy of the first excited state of $T_{n}$
as a criteria of convergence. 
In the standard setting,
after five Lanczos step,
we diagonalize $T_{n}$ every two Lanczos step.
If the energy of the first excited states coincides with
the previous energy within the specified accuracy,
the Lanczos iteration finishes.
The accuracy of the convergence can be specified by 
$\verb|CDataFileHead|$~(ModPara file in the expert mode).

After obtaining the eigenvalues,
we again perform the Lanczos iteration
to obtain the eigenvector.
From the eigenvectors $|n\rangle$,
we calculate 
energy $E_{n}=\langle n|\Ham|n\rangle $ and
variance $\Delta=\langle n|\Ham^{2}|n\rangle -(\langle n|\Ham|n\rangle)^2$.
If $E_{n}$ coincides with the eigenvalues obtained by the Lanczos iteration and 
$\Delta$ is smaller than the specified value,
we finish diagonalization.

If the accuracy of Lanczos method is not enough,
we perform the CG method to obtain the eigenvector.
As an initial vector of the CG method,
we use the eigenvectors obtained by the Lanczos method in the standard setting.
This often accelerates the convergence.
%----------------------------------------------------------
\section{Full Diagonalization method}
\label{Ch:AllDiagonalization}
\subsection{Over view}
We generate matrix of ${\hat{H}}$ by using the real space configuration 
$| \psi_j \rangle$($j=1\cdots d_{\rm H}$, $d_{\rm H}$ is dimension of the Hilbert space): 
$H_{ij}= \langle \psi_i | {\hat H} | \psi_j \rangle$.
By diagonalizing this matrix,
we can obtain all the eigenvalues $E_{i}$ and eigenvectors $|\Phi_i\rangle$ ($i=1 \cdots d_{\rm H}$). 
In the diagonalization, we use lapack routine such as \verb|dsyev| or \verb|zheev|.
We also calculate and out put
the expectation values $\la A_i\ra \equiv \langle \Phi_i | {\hat A} | \Phi_i\rangle$.
These values are used for the finite-temperature calculations.

\subsection{Finite-temperature calculations}
From
$\la A_i\ra \equiv \langle \Phi_i | {\hat A} | \Phi_i\rangle$,
we calculate finite-temperature properties by using the relation 
\begin{equation}
\langle {\hat A}\rangle=\frac{\sum_{i=1}^N \la A_i\ra {\rm  e}^{-\beta E_i}}{\sum_{i=1}^N{\rm  e}^{-\beta E_i}}.
\end{equation}
In the actual calculation are performed as the post scripts.

%----------------------------------------------------------
\section{Finite-temperature calculations by TPQ method}
\label{Ch:TPQ}
%----------------------------------------------------------
Sugiura and Shimizu show that
it is possible to calculate the finite-temperature properties
from a few wavefunctions (in the thermodynamic limit, only one wave function is necessary)~\cite{Sugiura2012}.
The wavefunction is called thermal pure quantum (TPQ) state.
Because TPQ state can be generated by operating the Hamiltonian 
to the random initial wavefunction,
we directly use the routine Lanczos method to the TPQ calculations.
Here, we explain how to construct micro canonical TPQ (mTPQ) state,
which offers the simplest way for finite-temperature calculations.

Let $|\psi_{0}\rangle$ a random initial vector.
By operating $(l-\hat{H}/N_{s})^{k}$($l$ is constant、
$N_{s}$ represents number of sites) 
to $|\psi_{0}\rangle$,
we obtain the $k$th TPQ states as
\begin{align}
|\psi_{k}\rangle \equiv \frac{(l-\hat{H}/N_{s})|\psi_{k-1}\rangle}{|(l-\hat{H}/N_{s})|\psi_{k-1}\rangle|}.
\end{align}
From  $|\psi_{k}\rangle$, we estimate corresponding inverse temperature $\beta_{k}$ as
\begin{align}
\beta_{k}\sim \frac{2k/N_{s}}{l-u_{k}},~~
u_{k} = \langle \psi_{k}|\hat{H}|\psi_{k}\rangle/N_{s},
\end{align}
where $u_{k}$ is the internal energy.
Arbitrary local physical properties at $\beta_{k}$ is also estimated as
\begin{align}
\langle \hat{A}\rangle_{\beta_{k}} =  \langle \psi_{k}|\hat{A}|\psi_{k}\rangle/N_{s}.
\end{align}

In finite-size system,
error is caused by the choice of the initial random vector.
To estimate the average value and error of the physical properties,
we perform some independent calculations by changing $|\psi_{0}\rangle$.
%----------------------------------------------------------
