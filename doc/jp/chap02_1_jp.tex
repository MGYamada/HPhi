
\begin{enumerate}

\item $\HPhi$ は次の場所からダウンロードできます。\\
\url{https://github.com/QLMS/HPhi/releases}

\item ダウンロードしたファイルを展開し、
展開したディレクトリのなかにある\verb|src/|ディレクトリに移動してください。
\begin{verbatim}
$ tar xzvf 
$ cd /src/
\end{verbatim}

\item インストール環境を設定します。
  ファイル\verb|make.sys|を編集してコンパイラやオプションの指定をしてください。
  いくつかの既知のシステムに対応する設定例がコメントとして\verb|make.sys|に記述されていますので
  それを用いることもできます。例えば、物性研究所システムB(sekirei)にインストールする場合には
  \verb|#####  ISSP System-B "sekirei" #####|と書いてある行の下にある
\begin{verbatim}
# CC = icc
# LAPACK_FLAGS = -Dlapack -mkl=parallel 
# FLAGS = -qopenmp  -O3 -xCORE-AVX2 -mcmodel=large -shared-intel
# MTFLAGS = -DDSFMT_MEXP=19937 $(FLAGS)
# INCLUDE_DIR=./include
\end{verbatim}
の行頭の\verb|#|を消してコメント解除をしてください。

\item 編集した\verb|make.sys|を保存し、次のように\verb|make|コマンドを実行してください。
\begin{verbatim}
$ make all
\end{verbatim}

これで実行可能ファイル\verb|HPhi|が生成されるので、
このディレクトリにパスを通すか、
パスの通っている場所にシンボリックリンクを作ってください。

\begin{screen}
\Large 
{\bf Tips}
\normalsize

実行ファイルにパスを通す時には、次のようにします。
\\
\verb|$ export PATH=${PATH}:|\underline{HPhiのディレクトリ}\verb|/src/|
\\
この設定を常に残すには、例えばログインシェルが\verb|bash|の場合には
\verb|~/.bashrc|ファイルに上記のコマンドを記載します。

\end{screen}

\end{enumerate}
