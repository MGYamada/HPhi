% !TEX root = userguide_jp.tex
%----------------------------------------------------------
$\HPhi$ は次の場所からダウンロードできます。\\
\url{https://github.com/QLMS/HPhi/releases}

ダウンロードしたファイルを次のように展開してください。
\begin{verbatim}
$ tar xzvf HPhi-xxx.tar.gz
\end{verbatim}

$\HPhi$は次の2通りの方法でインストールできます。

\subsection{\texttt{HPhiconfig.sh}を使う方法}

展開したディレクトリのなかにある\verb|HPhiconfig.sh|スクリプトを次のように実行してください。
(物性研システムB''sekirei''の場合)
\begin{verbatim}
$ bash HPhiconfig.sh sekirei
\end{verbatim}
これによりコンパイル環境設定ファイル\verb|make.sys|が\verb|src/|ディレクトリに作られます。
\verb|HPhiconfig.sh|の引数は次のものに対応しています。
\begin{itemize}
\item \verb|sekirei| : 物性研究所システムB ''sekirei''
\item \verb|maki| : 物性研究所システムC ''maki''
\item \verb|intel| : intelコンパイラ + Linux PC
\item \verb|mpicc-intel| : intelコンパイラ + MPIライブラリ(intelMPI以外) + Linux PC
\item \verb|gcc| : GCC + Linux PC
\item \verb|gcc-mac| : GCC + Mac
\end{itemize}

\verb|make.sys|の中身は次のようになっています(物性研システムB ''sekirei''の場合)。
\begin{verbatim}
 CC = mpiicc
 LAPACK_FLAGS = -Dlapack -mkl=parallel 
 FLAGS = -qopenmp  -O3 -xCORE-AVX2 -mcmodel=large -shared-intel -D MPI
 MTFLAGS = -DDSFMT_MEXP=19937 $(FLAGS)
 INCLUDE_DIR=./include
\end{verbatim}
となります。それぞれのマクロ(変数)の説明は次のとおりです。
\begin{itemize}
\item \verb|CC| : コンパイルコマンド(\verb|icc|, \verb|gcc|, \verb|fccpx|)
\item \verb|LAPACK_FLAGS| : LAPACKのためのコンパイルオプション。 \verb|-Dlapack|は必須です。
\item \verb|FLAGS| : その他のコンパイルオプション。
  \verb|-openmp|, \verb|-fopenmp|, \verb|-qopenmp|などのOpenMP有効化のオプションは必須です。
  MPI並列を行う場合は\verb|-D MPI|をつけます. 
\item \verb|MTFLAGS|, \verb|INCLUDE_DIR| : メルセンヌツイスター(乱数発生ルーチン)および
  追加のインクルードディレクトリを指定します。これらを変更する必要はありません。
\end{itemize}

これでコンパイルのための準備が整います。その後
\begin{verbatim}
$ make HPhi
\end{verbatim}
とすることで実行可能ファイル\verb|HPhi|が\verb|src/内に|生成されるので、
このディレクトリにパスを通すか、
パスの通っている場所にシンボリックリンクを作ってください。

\begin{screen}
\Large 
{\bf Tips}
\normalsize

実行ファイルにパスを通す時には、次のようにします。
\\
\verb|$ export PATH=${PATH}:|\underline{HPhiのディレクトリ}\verb|/src/|
\\
この設定を常に残すには、例えばログインシェルが\verb|bash|の場合には
\verb|~/.bashrc|ファイルに上記のコマンドを記載します。
\end{screen}

\subsection{cmakeを使う場合}
HPhiを展開したディレクトリのパスを\$PathTohphi 、ビルドディレクトリを\$HOME/build/hphi (任意の場所を指定可能)とした場合に、
\begin{verbatim}
cd $HOME/build/hphi
cmake -DCONFIG=gcc $PathTohphi
make
\end{verbatim}
でコンパイルすることができます。コンパイル後、\$HOME/build/hphi 直下にsrcフォルダが作成され、実行ファイルであるHPhiがそのフォルダ内に作成されます。MPIライブラリがない場合には、MPI非対応の実行ファイルが作成されます。

なお、上の例ではgccコンパイラを前提としたコンパイルになっていますが、
\begin{itemize}
\item \verb|sekirei| : 物性研究所システムB ''sekirei''
\item \verb|fujitsu| : 富士通コンパイラ(物性研究所システムC ''maki'')
\item \verb|intel| : intelコンパイラ + Linux PC
\item \verb|gcc| : GCC + Linux PC
\end{itemize}
のオプションが用意されています。以下、HPhiを展開したディレクトリでビルドする例を示します(intelコンパイラの場合)。
\begin{verbatim}
mkdir ./build
cd ./build
cmake -DCONFIG=intel ../
make
\end{verbatim}
実行後、buildフォルダ直下にsrcフォルダが作成され、HPhiがsrcフォルダ内に作成されます。なお、コンパイラを変更しコンパイルし直したい場合には、都度buildフォルダごと削除を行った上で、新規に上記作業を行うことをお薦めします。

